\documentclass[11pt]{article}

    \usepackage[breakable]{tcolorbox}
    \usepackage{parskip} % Stop auto-indenting (to mimic markdown behaviour)
    
    \usepackage{iftex}
    \ifPDFTeX
    	\usepackage[T1]{fontenc}
    	\usepackage{mathpazo}
    \else
    	\usepackage{fontspec}
    \fi

    % Basic figure setup, for now with no caption control since it's done
    % automatically by Pandoc (which extracts ![](path) syntax from Markdown).
    \usepackage{graphicx}
    % Maintain compatibility with old templates. Remove in nbconvert 6.0
    \let\Oldincludegraphics\includegraphics
    % Ensure that by default, figures have no caption (until we provide a
    % proper Figure object with a Caption API and a way to capture that
    % in the conversion process - todo).
    \usepackage{caption}
    \DeclareCaptionFormat{nocaption}{}
    \captionsetup{format=nocaption,aboveskip=0pt,belowskip=0pt}

    \usepackage[Export]{adjustbox} % Used to constrain images to a maximum size
    \adjustboxset{max size={0.9\linewidth}{0.9\paperheight}}
    \usepackage{float}
    \floatplacement{figure}{H} % forces figures to be placed at the correct location
    \usepackage{xcolor} % Allow colors to be defined
    \usepackage{enumerate} % Needed for markdown enumerations to work
    \usepackage{geometry} % Used to adjust the document margins
    \usepackage{amsmath} % Equations
    \usepackage{amssymb} % Equations
    \usepackage{textcomp} % defines textquotesingle
    % Hack from http://tex.stackexchange.com/a/47451/13684:
    \AtBeginDocument{%
        \def\PYZsq{\textquotesingle}% Upright quotes in Pygmentized code
    }
    \usepackage{upquote} % Upright quotes for verbatim code
    \usepackage{eurosym} % defines \euro
    \usepackage[mathletters]{ucs} % Extended unicode (utf-8) support
    \usepackage{fancyvrb} % verbatim replacement that allows latex
    \usepackage{grffile} % extends the file name processing of package graphics 
                         % to support a larger range
    \makeatletter % fix for grffile with XeLaTeX
    \def\Gread@@xetex#1{%
      \IfFileExists{"\Gin@base".bb}%
      {\Gread@eps{\Gin@base.bb}}%
      {\Gread@@xetex@aux#1}%
    }
    \makeatother

    % The hyperref package gives us a pdf with properly built
    % internal navigation ('pdf bookmarks' for the table of contents,
    % internal cross-reference links, web links for URLs, etc.)
    \usepackage{hyperref}
    % The default LaTeX title has an obnoxious amount of whitespace. By default,
    % titling removes some of it. It also provides customization options.
    \usepackage{titling}
    \usepackage{longtable} % longtable support required by pandoc >1.10
    \usepackage{booktabs}  % table support for pandoc > 1.12.2
    \usepackage[inline]{enumitem} % IRkernel/repr support (it uses the enumerate* environment)
    \usepackage[normalem]{ulem} % ulem is needed to support strikethroughs (\sout)
                                % normalem makes italics be italics, not underlines
    \usepackage{mathrsfs}
    

    
    % Colors for the hyperref package
    \definecolor{urlcolor}{rgb}{0,.145,.698}
    \definecolor{linkcolor}{rgb}{.71,0.21,0.01}
    \definecolor{citecolor}{rgb}{.12,.54,.11}

    % ANSI colors
    \definecolor{ansi-black}{HTML}{3E424D}
    \definecolor{ansi-black-intense}{HTML}{282C36}
    \definecolor{ansi-red}{HTML}{E75C58}
    \definecolor{ansi-red-intense}{HTML}{B22B31}
    \definecolor{ansi-green}{HTML}{00A250}
    \definecolor{ansi-green-intense}{HTML}{007427}
    \definecolor{ansi-yellow}{HTML}{DDB62B}
    \definecolor{ansi-yellow-intense}{HTML}{B27D12}
    \definecolor{ansi-blue}{HTML}{208FFB}
    \definecolor{ansi-blue-intense}{HTML}{0065CA}
    \definecolor{ansi-magenta}{HTML}{D160C4}
    \definecolor{ansi-magenta-intense}{HTML}{A03196}
    \definecolor{ansi-cyan}{HTML}{60C6C8}
    \definecolor{ansi-cyan-intense}{HTML}{258F8F}
    \definecolor{ansi-white}{HTML}{C5C1B4}
    \definecolor{ansi-white-intense}{HTML}{A1A6B2}
    \definecolor{ansi-default-inverse-fg}{HTML}{FFFFFF}
    \definecolor{ansi-default-inverse-bg}{HTML}{000000}

    % commands and environments needed by pandoc snippets
    % extracted from the output of `pandoc -s`
    \providecommand{\tightlist}{%
      \setlength{\itemsep}{0pt}\setlength{\parskip}{0pt}}
    \DefineVerbatimEnvironment{Highlighting}{Verbatim}{commandchars=\\\{\}}
    % Add ',fontsize=\small' for more characters per line
    \newenvironment{Shaded}{}{}
    \newcommand{\KeywordTok}[1]{\textcolor[rgb]{0.00,0.44,0.13}{\textbf{{#1}}}}
    \newcommand{\DataTypeTok}[1]{\textcolor[rgb]{0.56,0.13,0.00}{{#1}}}
    \newcommand{\DecValTok}[1]{\textcolor[rgb]{0.25,0.63,0.44}{{#1}}}
    \newcommand{\BaseNTok}[1]{\textcolor[rgb]{0.25,0.63,0.44}{{#1}}}
    \newcommand{\FloatTok}[1]{\textcolor[rgb]{0.25,0.63,0.44}{{#1}}}
    \newcommand{\CharTok}[1]{\textcolor[rgb]{0.25,0.44,0.63}{{#1}}}
    \newcommand{\StringTok}[1]{\textcolor[rgb]{0.25,0.44,0.63}{{#1}}}
    \newcommand{\CommentTok}[1]{\textcolor[rgb]{0.38,0.63,0.69}{\textit{{#1}}}}
    \newcommand{\OtherTok}[1]{\textcolor[rgb]{0.00,0.44,0.13}{{#1}}}
    \newcommand{\AlertTok}[1]{\textcolor[rgb]{1.00,0.00,0.00}{\textbf{{#1}}}}
    \newcommand{\FunctionTok}[1]{\textcolor[rgb]{0.02,0.16,0.49}{{#1}}}
    \newcommand{\RegionMarkerTok}[1]{{#1}}
    \newcommand{\ErrorTok}[1]{\textcolor[rgb]{1.00,0.00,0.00}{\textbf{{#1}}}}
    \newcommand{\NormalTok}[1]{{#1}}
    
    % Additional commands for more recent versions of Pandoc
    \newcommand{\ConstantTok}[1]{\textcolor[rgb]{0.53,0.00,0.00}{{#1}}}
    \newcommand{\SpecialCharTok}[1]{\textcolor[rgb]{0.25,0.44,0.63}{{#1}}}
    \newcommand{\VerbatimStringTok}[1]{\textcolor[rgb]{0.25,0.44,0.63}{{#1}}}
    \newcommand{\SpecialStringTok}[1]{\textcolor[rgb]{0.73,0.40,0.53}{{#1}}}
    \newcommand{\ImportTok}[1]{{#1}}
    \newcommand{\DocumentationTok}[1]{\textcolor[rgb]{0.73,0.13,0.13}{\textit{{#1}}}}
    \newcommand{\AnnotationTok}[1]{\textcolor[rgb]{0.38,0.63,0.69}{\textbf{\textit{{#1}}}}}
    \newcommand{\CommentVarTok}[1]{\textcolor[rgb]{0.38,0.63,0.69}{\textbf{\textit{{#1}}}}}
    \newcommand{\VariableTok}[1]{\textcolor[rgb]{0.10,0.09,0.49}{{#1}}}
    \newcommand{\ControlFlowTok}[1]{\textcolor[rgb]{0.00,0.44,0.13}{\textbf{{#1}}}}
    \newcommand{\OperatorTok}[1]{\textcolor[rgb]{0.40,0.40,0.40}{{#1}}}
    \newcommand{\BuiltInTok}[1]{{#1}}
    \newcommand{\ExtensionTok}[1]{{#1}}
    \newcommand{\PreprocessorTok}[1]{\textcolor[rgb]{0.74,0.48,0.00}{{#1}}}
    \newcommand{\AttributeTok}[1]{\textcolor[rgb]{0.49,0.56,0.16}{{#1}}}
    \newcommand{\InformationTok}[1]{\textcolor[rgb]{0.38,0.63,0.69}{\textbf{\textit{{#1}}}}}
    \newcommand{\WarningTok}[1]{\textcolor[rgb]{0.38,0.63,0.69}{\textbf{\textit{{#1}}}}}
    
    
    % Define a nice break command that doesn't care if a line doesn't already
    % exist.
    \def\br{\hspace*{\fill} \\* }
    % Math Jax compatibility definitions
    \def\gt{>}
    \def\lt{<}
    \let\Oldtex\TeX
    \let\Oldlatex\LaTeX
    \renewcommand{\TeX}{\textrm{\Oldtex}}
    \renewcommand{\LaTeX}{\textrm{\Oldlatex}}
    % Document parameters
    % Document title
    \title{end-to-end-classification-heart-diseas}
    
    
    
    
    
% Pygments definitions
\makeatletter
\def\PY@reset{\let\PY@it=\relax \let\PY@bf=\relax%
    \let\PY@ul=\relax \let\PY@tc=\relax%
    \let\PY@bc=\relax \let\PY@ff=\relax}
\def\PY@tok#1{\csname PY@tok@#1\endcsname}
\def\PY@toks#1+{\ifx\relax#1\empty\else%
    \PY@tok{#1}\expandafter\PY@toks\fi}
\def\PY@do#1{\PY@bc{\PY@tc{\PY@ul{%
    \PY@it{\PY@bf{\PY@ff{#1}}}}}}}
\def\PY#1#2{\PY@reset\PY@toks#1+\relax+\PY@do{#2}}

\expandafter\def\csname PY@tok@w\endcsname{\def\PY@tc##1{\textcolor[rgb]{0.73,0.73,0.73}{##1}}}
\expandafter\def\csname PY@tok@c\endcsname{\let\PY@it=\textit\def\PY@tc##1{\textcolor[rgb]{0.25,0.50,0.50}{##1}}}
\expandafter\def\csname PY@tok@cp\endcsname{\def\PY@tc##1{\textcolor[rgb]{0.74,0.48,0.00}{##1}}}
\expandafter\def\csname PY@tok@k\endcsname{\let\PY@bf=\textbf\def\PY@tc##1{\textcolor[rgb]{0.00,0.50,0.00}{##1}}}
\expandafter\def\csname PY@tok@kp\endcsname{\def\PY@tc##1{\textcolor[rgb]{0.00,0.50,0.00}{##1}}}
\expandafter\def\csname PY@tok@kt\endcsname{\def\PY@tc##1{\textcolor[rgb]{0.69,0.00,0.25}{##1}}}
\expandafter\def\csname PY@tok@o\endcsname{\def\PY@tc##1{\textcolor[rgb]{0.40,0.40,0.40}{##1}}}
\expandafter\def\csname PY@tok@ow\endcsname{\let\PY@bf=\textbf\def\PY@tc##1{\textcolor[rgb]{0.67,0.13,1.00}{##1}}}
\expandafter\def\csname PY@tok@nb\endcsname{\def\PY@tc##1{\textcolor[rgb]{0.00,0.50,0.00}{##1}}}
\expandafter\def\csname PY@tok@nf\endcsname{\def\PY@tc##1{\textcolor[rgb]{0.00,0.00,1.00}{##1}}}
\expandafter\def\csname PY@tok@nc\endcsname{\let\PY@bf=\textbf\def\PY@tc##1{\textcolor[rgb]{0.00,0.00,1.00}{##1}}}
\expandafter\def\csname PY@tok@nn\endcsname{\let\PY@bf=\textbf\def\PY@tc##1{\textcolor[rgb]{0.00,0.00,1.00}{##1}}}
\expandafter\def\csname PY@tok@ne\endcsname{\let\PY@bf=\textbf\def\PY@tc##1{\textcolor[rgb]{0.82,0.25,0.23}{##1}}}
\expandafter\def\csname PY@tok@nv\endcsname{\def\PY@tc##1{\textcolor[rgb]{0.10,0.09,0.49}{##1}}}
\expandafter\def\csname PY@tok@no\endcsname{\def\PY@tc##1{\textcolor[rgb]{0.53,0.00,0.00}{##1}}}
\expandafter\def\csname PY@tok@nl\endcsname{\def\PY@tc##1{\textcolor[rgb]{0.63,0.63,0.00}{##1}}}
\expandafter\def\csname PY@tok@ni\endcsname{\let\PY@bf=\textbf\def\PY@tc##1{\textcolor[rgb]{0.60,0.60,0.60}{##1}}}
\expandafter\def\csname PY@tok@na\endcsname{\def\PY@tc##1{\textcolor[rgb]{0.49,0.56,0.16}{##1}}}
\expandafter\def\csname PY@tok@nt\endcsname{\let\PY@bf=\textbf\def\PY@tc##1{\textcolor[rgb]{0.00,0.50,0.00}{##1}}}
\expandafter\def\csname PY@tok@nd\endcsname{\def\PY@tc##1{\textcolor[rgb]{0.67,0.13,1.00}{##1}}}
\expandafter\def\csname PY@tok@s\endcsname{\def\PY@tc##1{\textcolor[rgb]{0.73,0.13,0.13}{##1}}}
\expandafter\def\csname PY@tok@sd\endcsname{\let\PY@it=\textit\def\PY@tc##1{\textcolor[rgb]{0.73,0.13,0.13}{##1}}}
\expandafter\def\csname PY@tok@si\endcsname{\let\PY@bf=\textbf\def\PY@tc##1{\textcolor[rgb]{0.73,0.40,0.53}{##1}}}
\expandafter\def\csname PY@tok@se\endcsname{\let\PY@bf=\textbf\def\PY@tc##1{\textcolor[rgb]{0.73,0.40,0.13}{##1}}}
\expandafter\def\csname PY@tok@sr\endcsname{\def\PY@tc##1{\textcolor[rgb]{0.73,0.40,0.53}{##1}}}
\expandafter\def\csname PY@tok@ss\endcsname{\def\PY@tc##1{\textcolor[rgb]{0.10,0.09,0.49}{##1}}}
\expandafter\def\csname PY@tok@sx\endcsname{\def\PY@tc##1{\textcolor[rgb]{0.00,0.50,0.00}{##1}}}
\expandafter\def\csname PY@tok@m\endcsname{\def\PY@tc##1{\textcolor[rgb]{0.40,0.40,0.40}{##1}}}
\expandafter\def\csname PY@tok@gh\endcsname{\let\PY@bf=\textbf\def\PY@tc##1{\textcolor[rgb]{0.00,0.00,0.50}{##1}}}
\expandafter\def\csname PY@tok@gu\endcsname{\let\PY@bf=\textbf\def\PY@tc##1{\textcolor[rgb]{0.50,0.00,0.50}{##1}}}
\expandafter\def\csname PY@tok@gd\endcsname{\def\PY@tc##1{\textcolor[rgb]{0.63,0.00,0.00}{##1}}}
\expandafter\def\csname PY@tok@gi\endcsname{\def\PY@tc##1{\textcolor[rgb]{0.00,0.63,0.00}{##1}}}
\expandafter\def\csname PY@tok@gr\endcsname{\def\PY@tc##1{\textcolor[rgb]{1.00,0.00,0.00}{##1}}}
\expandafter\def\csname PY@tok@ge\endcsname{\let\PY@it=\textit}
\expandafter\def\csname PY@tok@gs\endcsname{\let\PY@bf=\textbf}
\expandafter\def\csname PY@tok@gp\endcsname{\let\PY@bf=\textbf\def\PY@tc##1{\textcolor[rgb]{0.00,0.00,0.50}{##1}}}
\expandafter\def\csname PY@tok@go\endcsname{\def\PY@tc##1{\textcolor[rgb]{0.53,0.53,0.53}{##1}}}
\expandafter\def\csname PY@tok@gt\endcsname{\def\PY@tc##1{\textcolor[rgb]{0.00,0.27,0.87}{##1}}}
\expandafter\def\csname PY@tok@err\endcsname{\def\PY@bc##1{\setlength{\fboxsep}{0pt}\fcolorbox[rgb]{1.00,0.00,0.00}{1,1,1}{\strut ##1}}}
\expandafter\def\csname PY@tok@kc\endcsname{\let\PY@bf=\textbf\def\PY@tc##1{\textcolor[rgb]{0.00,0.50,0.00}{##1}}}
\expandafter\def\csname PY@tok@kd\endcsname{\let\PY@bf=\textbf\def\PY@tc##1{\textcolor[rgb]{0.00,0.50,0.00}{##1}}}
\expandafter\def\csname PY@tok@kn\endcsname{\let\PY@bf=\textbf\def\PY@tc##1{\textcolor[rgb]{0.00,0.50,0.00}{##1}}}
\expandafter\def\csname PY@tok@kr\endcsname{\let\PY@bf=\textbf\def\PY@tc##1{\textcolor[rgb]{0.00,0.50,0.00}{##1}}}
\expandafter\def\csname PY@tok@bp\endcsname{\def\PY@tc##1{\textcolor[rgb]{0.00,0.50,0.00}{##1}}}
\expandafter\def\csname PY@tok@fm\endcsname{\def\PY@tc##1{\textcolor[rgb]{0.00,0.00,1.00}{##1}}}
\expandafter\def\csname PY@tok@vc\endcsname{\def\PY@tc##1{\textcolor[rgb]{0.10,0.09,0.49}{##1}}}
\expandafter\def\csname PY@tok@vg\endcsname{\def\PY@tc##1{\textcolor[rgb]{0.10,0.09,0.49}{##1}}}
\expandafter\def\csname PY@tok@vi\endcsname{\def\PY@tc##1{\textcolor[rgb]{0.10,0.09,0.49}{##1}}}
\expandafter\def\csname PY@tok@vm\endcsname{\def\PY@tc##1{\textcolor[rgb]{0.10,0.09,0.49}{##1}}}
\expandafter\def\csname PY@tok@sa\endcsname{\def\PY@tc##1{\textcolor[rgb]{0.73,0.13,0.13}{##1}}}
\expandafter\def\csname PY@tok@sb\endcsname{\def\PY@tc##1{\textcolor[rgb]{0.73,0.13,0.13}{##1}}}
\expandafter\def\csname PY@tok@sc\endcsname{\def\PY@tc##1{\textcolor[rgb]{0.73,0.13,0.13}{##1}}}
\expandafter\def\csname PY@tok@dl\endcsname{\def\PY@tc##1{\textcolor[rgb]{0.73,0.13,0.13}{##1}}}
\expandafter\def\csname PY@tok@s2\endcsname{\def\PY@tc##1{\textcolor[rgb]{0.73,0.13,0.13}{##1}}}
\expandafter\def\csname PY@tok@sh\endcsname{\def\PY@tc##1{\textcolor[rgb]{0.73,0.13,0.13}{##1}}}
\expandafter\def\csname PY@tok@s1\endcsname{\def\PY@tc##1{\textcolor[rgb]{0.73,0.13,0.13}{##1}}}
\expandafter\def\csname PY@tok@mb\endcsname{\def\PY@tc##1{\textcolor[rgb]{0.40,0.40,0.40}{##1}}}
\expandafter\def\csname PY@tok@mf\endcsname{\def\PY@tc##1{\textcolor[rgb]{0.40,0.40,0.40}{##1}}}
\expandafter\def\csname PY@tok@mh\endcsname{\def\PY@tc##1{\textcolor[rgb]{0.40,0.40,0.40}{##1}}}
\expandafter\def\csname PY@tok@mi\endcsname{\def\PY@tc##1{\textcolor[rgb]{0.40,0.40,0.40}{##1}}}
\expandafter\def\csname PY@tok@il\endcsname{\def\PY@tc##1{\textcolor[rgb]{0.40,0.40,0.40}{##1}}}
\expandafter\def\csname PY@tok@mo\endcsname{\def\PY@tc##1{\textcolor[rgb]{0.40,0.40,0.40}{##1}}}
\expandafter\def\csname PY@tok@ch\endcsname{\let\PY@it=\textit\def\PY@tc##1{\textcolor[rgb]{0.25,0.50,0.50}{##1}}}
\expandafter\def\csname PY@tok@cm\endcsname{\let\PY@it=\textit\def\PY@tc##1{\textcolor[rgb]{0.25,0.50,0.50}{##1}}}
\expandafter\def\csname PY@tok@cpf\endcsname{\let\PY@it=\textit\def\PY@tc##1{\textcolor[rgb]{0.25,0.50,0.50}{##1}}}
\expandafter\def\csname PY@tok@c1\endcsname{\let\PY@it=\textit\def\PY@tc##1{\textcolor[rgb]{0.25,0.50,0.50}{##1}}}
\expandafter\def\csname PY@tok@cs\endcsname{\let\PY@it=\textit\def\PY@tc##1{\textcolor[rgb]{0.25,0.50,0.50}{##1}}}

\def\PYZbs{\char`\\}
\def\PYZus{\char`\_}
\def\PYZob{\char`\{}
\def\PYZcb{\char`\}}
\def\PYZca{\char`\^}
\def\PYZam{\char`\&}
\def\PYZlt{\char`\<}
\def\PYZgt{\char`\>}
\def\PYZsh{\char`\#}
\def\PYZpc{\char`\%}
\def\PYZdl{\char`\$}
\def\PYZhy{\char`\-}
\def\PYZsq{\char`\'}
\def\PYZdq{\char`\"}
\def\PYZti{\char`\~}
% for compatibility with earlier versions
\def\PYZat{@}
\def\PYZlb{[}
\def\PYZrb{]}
\makeatother


    % For linebreaks inside Verbatim environment from package fancyvrb. 
    \makeatletter
        \newbox\Wrappedcontinuationbox 
        \newbox\Wrappedvisiblespacebox 
        \newcommand*\Wrappedvisiblespace {\textcolor{red}{\textvisiblespace}} 
        \newcommand*\Wrappedcontinuationsymbol {\textcolor{red}{\llap{\tiny$\m@th\hookrightarrow$}}} 
        \newcommand*\Wrappedcontinuationindent {3ex } 
        \newcommand*\Wrappedafterbreak {\kern\Wrappedcontinuationindent\copy\Wrappedcontinuationbox} 
        % Take advantage of the already applied Pygments mark-up to insert 
        % potential linebreaks for TeX processing. 
        %        {, <, #, %, $, ' and ": go to next line. 
        %        _, }, ^, &, >, - and ~: stay at end of broken line. 
        % Use of \textquotesingle for straight quote. 
        \newcommand*\Wrappedbreaksatspecials {% 
            \def\PYGZus{\discretionary{\char`\_}{\Wrappedafterbreak}{\char`\_}}% 
            \def\PYGZob{\discretionary{}{\Wrappedafterbreak\char`\{}{\char`\{}}% 
            \def\PYGZcb{\discretionary{\char`\}}{\Wrappedafterbreak}{\char`\}}}% 
            \def\PYGZca{\discretionary{\char`\^}{\Wrappedafterbreak}{\char`\^}}% 
            \def\PYGZam{\discretionary{\char`\&}{\Wrappedafterbreak}{\char`\&}}% 
            \def\PYGZlt{\discretionary{}{\Wrappedafterbreak\char`\<}{\char`\<}}% 
            \def\PYGZgt{\discretionary{\char`\>}{\Wrappedafterbreak}{\char`\>}}% 
            \def\PYGZsh{\discretionary{}{\Wrappedafterbreak\char`\#}{\char`\#}}% 
            \def\PYGZpc{\discretionary{}{\Wrappedafterbreak\char`\%}{\char`\%}}% 
            \def\PYGZdl{\discretionary{}{\Wrappedafterbreak\char`\$}{\char`\$}}% 
            \def\PYGZhy{\discretionary{\char`\-}{\Wrappedafterbreak}{\char`\-}}% 
            \def\PYGZsq{\discretionary{}{\Wrappedafterbreak\textquotesingle}{\textquotesingle}}% 
            \def\PYGZdq{\discretionary{}{\Wrappedafterbreak\char`\"}{\char`\"}}% 
            \def\PYGZti{\discretionary{\char`\~}{\Wrappedafterbreak}{\char`\~}}% 
        } 
        % Some characters . , ; ? ! / are not pygmentized. 
        % This macro makes them "active" and they will insert potential linebreaks 
        \newcommand*\Wrappedbreaksatpunct {% 
            \lccode`\~`\.\lowercase{\def~}{\discretionary{\hbox{\char`\.}}{\Wrappedafterbreak}{\hbox{\char`\.}}}% 
            \lccode`\~`\,\lowercase{\def~}{\discretionary{\hbox{\char`\,}}{\Wrappedafterbreak}{\hbox{\char`\,}}}% 
            \lccode`\~`\;\lowercase{\def~}{\discretionary{\hbox{\char`\;}}{\Wrappedafterbreak}{\hbox{\char`\;}}}% 
            \lccode`\~`\:\lowercase{\def~}{\discretionary{\hbox{\char`\:}}{\Wrappedafterbreak}{\hbox{\char`\:}}}% 
            \lccode`\~`\?\lowercase{\def~}{\discretionary{\hbox{\char`\?}}{\Wrappedafterbreak}{\hbox{\char`\?}}}% 
            \lccode`\~`\!\lowercase{\def~}{\discretionary{\hbox{\char`\!}}{\Wrappedafterbreak}{\hbox{\char`\!}}}% 
            \lccode`\~`\/\lowercase{\def~}{\discretionary{\hbox{\char`\/}}{\Wrappedafterbreak}{\hbox{\char`\/}}}% 
            \catcode`\.\active
            \catcode`\,\active 
            \catcode`\;\active
            \catcode`\:\active
            \catcode`\?\active
            \catcode`\!\active
            \catcode`\/\active 
            \lccode`\~`\~ 	
        }
    \makeatother

    \let\OriginalVerbatim=\Verbatim
    \makeatletter
    \renewcommand{\Verbatim}[1][1]{%
        %\parskip\z@skip
        \sbox\Wrappedcontinuationbox {\Wrappedcontinuationsymbol}%
        \sbox\Wrappedvisiblespacebox {\FV@SetupFont\Wrappedvisiblespace}%
        \def\FancyVerbFormatLine ##1{\hsize\linewidth
            \vtop{\raggedright\hyphenpenalty\z@\exhyphenpenalty\z@
                \doublehyphendemerits\z@\finalhyphendemerits\z@
                \strut ##1\strut}%
        }%
        % If the linebreak is at a space, the latter will be displayed as visible
        % space at end of first line, and a continuation symbol starts next line.
        % Stretch/shrink are however usually zero for typewriter font.
        \def\FV@Space {%
            \nobreak\hskip\z@ plus\fontdimen3\font minus\fontdimen4\font
            \discretionary{\copy\Wrappedvisiblespacebox}{\Wrappedafterbreak}
            {\kern\fontdimen2\font}%
        }%
        
        % Allow breaks at special characters using \PYG... macros.
        \Wrappedbreaksatspecials
        % Breaks at punctuation characters . , ; ? ! and / need catcode=\active 	
        \OriginalVerbatim[#1,codes*=\Wrappedbreaksatpunct]%
    }
    \makeatother

    % Exact colors from NB
    \definecolor{incolor}{HTML}{303F9F}
    \definecolor{outcolor}{HTML}{D84315}
    \definecolor{cellborder}{HTML}{CFCFCF}
    \definecolor{cellbackground}{HTML}{F7F7F7}
    
    % prompt
    \makeatletter
    \newcommand{\boxspacing}{\kern\kvtcb@left@rule\kern\kvtcb@boxsep}
    \makeatother
    \newcommand{\prompt}[4]{
        \ttfamily\llap{{\color{#2}[#3]:\hspace{3pt}#4}}\vspace{-\baselineskip}
    }
    

    
    % Prevent overflowing lines due to hard-to-break entities
    \sloppy 
    % Setup hyperref package
    \hypersetup{
      breaklinks=true,  % so long urls are correctly broken across lines
      colorlinks=true,
      urlcolor=urlcolor,
      linkcolor=linkcolor,
      citecolor=citecolor,
      }
    % Slightly bigger margins than the latex defaults
    
    \geometry{verbose,tmargin=1in,bmargin=1in,lmargin=1in,rmargin=1in}
    
    

\begin{document}
    
    \maketitle
    
    

    
    \hypertarget{predicting-heart-disease-using-machine-learning}{%
\section{Predicting heart disease using machine
learning}\label{predicting-heart-disease-using-machine-learning}}

This notebook looks into using varius Python-based machine learning and
data science libraries in an attempt to build a machine learning model
capable of predicting weather or not someone has heart disease based on
their medical attributes

We are going to take the following aproach : 1. Problem definition 2.
Data 3. Evaluation 4. Features 5. Modelling 6. Experimentation

    \hypertarget{problem-definition}{%
\subsection{1. Problem Definition}\label{problem-definition}}

\begin{quote}
Given clinical parameters about a patient, can we predict weather or not
they have heart disease?
\end{quote}

    \hypertarget{data}{%
\subsection{2. Data}\label{data}}

The original data came from the Cleaveland data from the UCI machine
learning repository
https://archive.ics.uci.edu/ml/datasets/heart+Disease, or kaggle
https://www.kaggle.com/ronitf/heart-disease-uci/data?select=heart.csv

    \hypertarget{evaluation}{%
\subsection{3. Evaluation}\label{evaluation}}

\begin{quote}
if we can reach 95\% accuracy at predicting weather or not a patient has
a heart disease during the proof of concept, we'll pursue the project
\end{quote}

    \hypertarget{features}{%
\subsection{4.Features}\label{features}}

this is where you'll get different information about eavh of he features
in your data. \textbf{Create a data dictionary}

\#3 (age) \#4 (sex) \#9 (cp) \#10 (trestbps) \#12 (chol) \#16 (fbs) \#19
(restecg) \#32 (thalach) \#38 (exang) \#40 (oldpeak) \#41 (slope) \#44
(ca) \#51 (thal) \#58 (num) (the predicted attribute)

    \hypertarget{preparing-the-tools}{%
\subsection{Preparing the tools}\label{preparing-the-tools}}

Import all the tools

    \begin{tcolorbox}[breakable, size=fbox, boxrule=1pt, pad at break*=1mm,colback=cellbackground, colframe=cellborder]
\prompt{In}{incolor}{1}{\boxspacing}
\begin{Verbatim}[commandchars=\\\{\}]
\PY{c+c1}{\PYZsh{}import libraries }

\PY{c+c1}{\PYZsh{}Regular EDA(explratory data analysis) and ploting libraries }
\PY{k+kn}{import} \PY{n+nn}{numpy} \PY{k}{as} \PY{n+nn}{np} 
\PY{k+kn}{import} \PY{n+nn}{pandas} \PY{k}{as} \PY{n+nn}{pd} 
\PY{k+kn}{import} \PY{n+nn}{matplotlib}\PY{n+nn}{.}\PY{n+nn}{pyplot} \PY{k}{as} \PY{n+nn}{plt} 
\PY{o}{\PYZpc{}}\PY{k}{matplotlib} inline 
\PY{k+kn}{import} \PY{n+nn}{sklearn} 
\PY{k+kn}{import} \PY{n+nn}{seaborn} \PY{k}{as} \PY{n+nn}{sns} 

\PY{c+c1}{\PYZsh{}models form scikit\PYZhy{}learn }
\PY{k+kn}{from} \PY{n+nn}{sklearn}\PY{n+nn}{.}\PY{n+nn}{linear\PYZus{}model} \PY{k+kn}{import} \PY{n}{LogisticRegression}
\PY{k+kn}{from} \PY{n+nn}{sklearn}\PY{n+nn}{.}\PY{n+nn}{neighbors} \PY{k+kn}{import} \PY{n}{KNeighborsClassifier}
\PY{k+kn}{from} \PY{n+nn}{sklearn}\PY{n+nn}{.}\PY{n+nn}{ensemble} \PY{k+kn}{import} \PY{n}{RandomForestClassifier} 

\PY{c+c1}{\PYZsh{}Model Evaluation }
\PY{k+kn}{from} \PY{n+nn}{sklearn}\PY{n+nn}{.}\PY{n+nn}{model\PYZus{}selection} \PY{k+kn}{import} \PY{n}{train\PYZus{}test\PYZus{}split}\PY{p}{,} \PY{n}{cross\PYZus{}val\PYZus{}score}
\PY{k+kn}{from} \PY{n+nn}{sklearn}\PY{n+nn}{.}\PY{n+nn}{model\PYZus{}selection} \PY{k+kn}{import} \PY{n}{RandomizedSearchCV}\PY{p}{,} \PY{n}{GridSearchCV}
\PY{k+kn}{from} \PY{n+nn}{sklearn}\PY{n+nn}{.}\PY{n+nn}{metrics} \PY{k+kn}{import} \PY{n}{confusion\PYZus{}matrix}\PY{p}{,} \PY{n}{classification\PYZus{}report} 
\PY{k+kn}{from} \PY{n+nn}{sklearn}\PY{n+nn}{.}\PY{n+nn}{metrics} \PY{k+kn}{import} \PY{n}{precision\PYZus{}score}\PY{p}{,} \PY{n}{recall\PYZus{}score}\PY{p}{,} \PY{n}{f1\PYZus{}score} 
\PY{k+kn}{from} \PY{n+nn}{sklearn}\PY{n+nn}{.}\PY{n+nn}{metrics} \PY{k+kn}{import} \PY{n}{plot\PYZus{}roc\PYZus{}curve} 
\end{Verbatim}
\end{tcolorbox}

    \begin{tcolorbox}[breakable, size=fbox, boxrule=1pt, pad at break*=1mm,colback=cellbackground, colframe=cellborder]
\prompt{In}{incolor}{2}{\boxspacing}
\begin{Verbatim}[commandchars=\\\{\}]
\PY{c+c1}{\PYZsh{} import sys}
\PY{c+c1}{\PYZsh{} !conda install \PYZhy{}\PYZhy{}yes \PYZhy{}\PYZhy{}prefix \PYZob{}sys.prefix\PYZcb{} seaborn}
\end{Verbatim}
\end{tcolorbox}

    \hypertarget{load-data}{%
\subsection{Load data}\label{load-data}}

    \begin{tcolorbox}[breakable, size=fbox, boxrule=1pt, pad at break*=1mm,colback=cellbackground, colframe=cellborder]
\prompt{In}{incolor}{3}{\boxspacing}
\begin{Verbatim}[commandchars=\\\{\}]
\PY{n}{df} \PY{o}{=} \PY{n}{pd}\PY{o}{.}\PY{n}{read\PYZus{}csv}\PY{p}{(}\PY{l+s+s1}{\PYZsq{}}\PY{l+s+s1}{Data/heart\PYZus{}disease.csv}\PY{l+s+s1}{\PYZsq{}}\PY{p}{)}
\PY{n}{df}
\end{Verbatim}
\end{tcolorbox}

            \begin{tcolorbox}[breakable, size=fbox, boxrule=.5pt, pad at break*=1mm, opacityfill=0]
\prompt{Out}{outcolor}{3}{\boxspacing}
\begin{Verbatim}[commandchars=\\\{\}]
     age  sex  cp  trestbps  chol  fbs  restecg  thalach  exang  oldpeak  \textbackslash{}
0     63    1   3       145   233    1        0      150      0      2.3
1     37    1   2       130   250    0        1      187      0      3.5
2     41    0   1       130   204    0        0      172      0      1.4
3     56    1   1       120   236    0        1      178      0      0.8
4     57    0   0       120   354    0        1      163      1      0.6
..   {\ldots}  {\ldots}  ..       {\ldots}   {\ldots}  {\ldots}      {\ldots}      {\ldots}    {\ldots}      {\ldots}
298   57    0   0       140   241    0        1      123      1      0.2
299   45    1   3       110   264    0        1      132      0      1.2
300   68    1   0       144   193    1        1      141      0      3.4
301   57    1   0       130   131    0        1      115      1      1.2
302   57    0   1       130   236    0        0      174      0      0.0

     slope  ca  thal  target
0        0   0     1       1
1        0   0     2       1
2        2   0     2       1
3        2   0     2       1
4        2   0     2       1
..     {\ldots}  ..   {\ldots}     {\ldots}
298      1   0     3       0
299      1   0     3       0
300      1   2     3       0
301      1   1     3       0
302      1   1     2       0

[303 rows x 14 columns]
\end{Verbatim}
\end{tcolorbox}
        
    \begin{tcolorbox}[breakable, size=fbox, boxrule=1pt, pad at break*=1mm,colback=cellbackground, colframe=cellborder]
\prompt{In}{incolor}{4}{\boxspacing}
\begin{Verbatim}[commandchars=\\\{\}]
\PY{n}{df}\PY{o}{.}\PY{n}{shape} \PY{c+c1}{\PYZsh{}(rows, columns )}
\end{Verbatim}
\end{tcolorbox}

            \begin{tcolorbox}[breakable, size=fbox, boxrule=.5pt, pad at break*=1mm, opacityfill=0]
\prompt{Out}{outcolor}{4}{\boxspacing}
\begin{Verbatim}[commandchars=\\\{\}]
(303, 14)
\end{Verbatim}
\end{tcolorbox}
        
    \hypertarget{data-exploration}{%
\subsection{Data exploration}\label{data-exploration}}

the goal here is to become an expert in the data you are working with

\begin{enumerate}
\def\labelenumi{\arabic{enumi}.}
\tightlist
\item
  What questions are you trying to solve
\item
  what kind of data do we have and how do we treat different types?
\item
  what is missing from the data and how do we deal with it
\item
  where are the outliers and why should we care about it?
\item
  How can you add, change or remove features to get mpre out of ypur
  data?
\end{enumerate}

    \begin{tcolorbox}[breakable, size=fbox, boxrule=1pt, pad at break*=1mm,colback=cellbackground, colframe=cellborder]
\prompt{In}{incolor}{5}{\boxspacing}
\begin{Verbatim}[commandchars=\\\{\}]
\PY{n}{df}\PY{o}{.}\PY{n}{head}\PY{p}{(}\PY{p}{)}
\end{Verbatim}
\end{tcolorbox}

            \begin{tcolorbox}[breakable, size=fbox, boxrule=.5pt, pad at break*=1mm, opacityfill=0]
\prompt{Out}{outcolor}{5}{\boxspacing}
\begin{Verbatim}[commandchars=\\\{\}]
   age  sex  cp  trestbps  chol  fbs  restecg  thalach  exang  oldpeak  slope  \textbackslash{}
0   63    1   3       145   233    1        0      150      0      2.3      0
1   37    1   2       130   250    0        1      187      0      3.5      0
2   41    0   1       130   204    0        0      172      0      1.4      2
3   56    1   1       120   236    0        1      178      0      0.8      2
4   57    0   0       120   354    0        1      163      1      0.6      2

   ca  thal  target
0   0     1       1
1   0     2       1
2   0     2       1
3   0     2       1
4   0     2       1
\end{Verbatim}
\end{tcolorbox}
        
    \begin{tcolorbox}[breakable, size=fbox, boxrule=1pt, pad at break*=1mm,colback=cellbackground, colframe=cellborder]
\prompt{In}{incolor}{6}{\boxspacing}
\begin{Verbatim}[commandchars=\\\{\}]
\PY{n}{df}\PY{o}{.}\PY{n}{tail}\PY{p}{(}\PY{p}{)}
\end{Verbatim}
\end{tcolorbox}

            \begin{tcolorbox}[breakable, size=fbox, boxrule=.5pt, pad at break*=1mm, opacityfill=0]
\prompt{Out}{outcolor}{6}{\boxspacing}
\begin{Verbatim}[commandchars=\\\{\}]
     age  sex  cp  trestbps  chol  fbs  restecg  thalach  exang  oldpeak  \textbackslash{}
298   57    0   0       140   241    0        1      123      1      0.2
299   45    1   3       110   264    0        1      132      0      1.2
300   68    1   0       144   193    1        1      141      0      3.4
301   57    1   0       130   131    0        1      115      1      1.2
302   57    0   1       130   236    0        0      174      0      0.0

     slope  ca  thal  target
298      1   0     3       0
299      1   0     3       0
300      1   2     3       0
301      1   1     3       0
302      1   1     2       0
\end{Verbatim}
\end{tcolorbox}
        
    \begin{tcolorbox}[breakable, size=fbox, boxrule=1pt, pad at break*=1mm,colback=cellbackground, colframe=cellborder]
\prompt{In}{incolor}{7}{\boxspacing}
\begin{Verbatim}[commandchars=\\\{\}]
\PY{c+c1}{\PYZsh{}lest find out how many of each class there is in our data set}
\PY{n}{df}\PY{o}{.}\PY{n}{target}\PY{o}{.}\PY{n}{value\PYZus{}counts}\PY{p}{(}\PY{p}{)}
\end{Verbatim}
\end{tcolorbox}

            \begin{tcolorbox}[breakable, size=fbox, boxrule=.5pt, pad at break*=1mm, opacityfill=0]
\prompt{Out}{outcolor}{7}{\boxspacing}
\begin{Verbatim}[commandchars=\\\{\}]
1    165
0    138
Name: target, dtype: int64
\end{Verbatim}
\end{tcolorbox}
        
    \begin{tcolorbox}[breakable, size=fbox, boxrule=1pt, pad at break*=1mm,colback=cellbackground, colframe=cellborder]
\prompt{In}{incolor}{8}{\boxspacing}
\begin{Verbatim}[commandchars=\\\{\}]
\PY{n}{df}\PY{p}{[}\PY{l+s+s2}{\PYZdq{}}\PY{l+s+s2}{target}\PY{l+s+s2}{\PYZdq{}}\PY{p}{]}\PY{o}{.}\PY{n}{value\PYZus{}counts}\PY{p}{(}\PY{p}{)}\PY{o}{.}\PY{n}{plot}\PY{p}{(}\PY{n}{kind}\PY{o}{=}\PY{l+s+s2}{\PYZdq{}}\PY{l+s+s2}{bar}\PY{l+s+s2}{\PYZdq{}}\PY{p}{,} \PY{n}{color}\PY{o}{=}\PY{p}{[}\PY{l+s+s2}{\PYZdq{}}\PY{l+s+s2}{salmon}\PY{l+s+s2}{\PYZdq{}}\PY{p}{,} \PY{l+s+s2}{\PYZdq{}}\PY{l+s+s2}{lightblue}\PY{l+s+s2}{\PYZdq{}}\PY{p}{]}\PY{p}{)}\PY{p}{;}
\end{Verbatim}
\end{tcolorbox}

    \begin{center}
    \adjustimage{max size={0.9\linewidth}{0.9\paperheight}}{output_15_0.png}
    \end{center}
    { \hspace*{\fill} \\}
    
    \begin{tcolorbox}[breakable, size=fbox, boxrule=1pt, pad at break*=1mm,colback=cellbackground, colframe=cellborder]
\prompt{In}{incolor}{9}{\boxspacing}
\begin{Verbatim}[commandchars=\\\{\}]
\PY{n}{df}\PY{o}{.}\PY{n}{info}\PY{p}{(}\PY{p}{)}
\end{Verbatim}
\end{tcolorbox}

    \begin{Verbatim}[commandchars=\\\{\}]
<class 'pandas.core.frame.DataFrame'>
RangeIndex: 303 entries, 0 to 302
Data columns (total 14 columns):
 \#   Column    Non-Null Count  Dtype
---  ------    --------------  -----
 0   age       303 non-null    int64
 1   sex       303 non-null    int64
 2   cp        303 non-null    int64
 3   trestbps  303 non-null    int64
 4   chol      303 non-null    int64
 5   fbs       303 non-null    int64
 6   restecg   303 non-null    int64
 7   thalach   303 non-null    int64
 8   exang     303 non-null    int64
 9   oldpeak   303 non-null    float64
 10  slope     303 non-null    int64
 11  ca        303 non-null    int64
 12  thal      303 non-null    int64
 13  target    303 non-null    int64
dtypes: float64(1), int64(13)
memory usage: 33.3 KB
    \end{Verbatim}

    \begin{tcolorbox}[breakable, size=fbox, boxrule=1pt, pad at break*=1mm,colback=cellbackground, colframe=cellborder]
\prompt{In}{incolor}{10}{\boxspacing}
\begin{Verbatim}[commandchars=\\\{\}]
\PY{c+c1}{\PYZsh{}are there any missing values?}
\PY{n}{df}\PY{o}{.}\PY{n}{isna}\PY{p}{(}\PY{p}{)}\PY{o}{.}\PY{n}{sum}\PY{p}{(}\PY{p}{)}
\end{Verbatim}
\end{tcolorbox}

            \begin{tcolorbox}[breakable, size=fbox, boxrule=.5pt, pad at break*=1mm, opacityfill=0]
\prompt{Out}{outcolor}{10}{\boxspacing}
\begin{Verbatim}[commandchars=\\\{\}]
age         0
sex         0
cp          0
trestbps    0
chol        0
fbs         0
restecg     0
thalach     0
exang       0
oldpeak     0
slope       0
ca          0
thal        0
target      0
dtype: int64
\end{Verbatim}
\end{tcolorbox}
        
    \begin{tcolorbox}[breakable, size=fbox, boxrule=1pt, pad at break*=1mm,colback=cellbackground, colframe=cellborder]
\prompt{In}{incolor}{11}{\boxspacing}
\begin{Verbatim}[commandchars=\\\{\}]
\PY{n}{df}\PY{o}{.}\PY{n}{describe}\PY{p}{(}\PY{p}{)}
\end{Verbatim}
\end{tcolorbox}

            \begin{tcolorbox}[breakable, size=fbox, boxrule=.5pt, pad at break*=1mm, opacityfill=0]
\prompt{Out}{outcolor}{11}{\boxspacing}
\begin{Verbatim}[commandchars=\\\{\}]
              age         sex          cp    trestbps        chol         fbs  \textbackslash{}
count  303.000000  303.000000  303.000000  303.000000  303.000000  303.000000
mean    54.366337    0.683168    0.966997  131.623762  246.264026    0.148515
std      9.082101    0.466011    1.032052   17.538143   51.830751    0.356198
min     29.000000    0.000000    0.000000   94.000000  126.000000    0.000000
25\%     47.500000    0.000000    0.000000  120.000000  211.000000    0.000000
50\%     55.000000    1.000000    1.000000  130.000000  240.000000    0.000000
75\%     61.000000    1.000000    2.000000  140.000000  274.500000    0.000000
max     77.000000    1.000000    3.000000  200.000000  564.000000    1.000000

          restecg     thalach       exang     oldpeak       slope          ca  \textbackslash{}
count  303.000000  303.000000  303.000000  303.000000  303.000000  303.000000
mean     0.528053  149.646865    0.326733    1.039604    1.399340    0.729373
std      0.525860   22.905161    0.469794    1.161075    0.616226    1.022606
min      0.000000   71.000000    0.000000    0.000000    0.000000    0.000000
25\%      0.000000  133.500000    0.000000    0.000000    1.000000    0.000000
50\%      1.000000  153.000000    0.000000    0.800000    1.000000    0.000000
75\%      1.000000  166.000000    1.000000    1.600000    2.000000    1.000000
max      2.000000  202.000000    1.000000    6.200000    2.000000    4.000000

             thal      target
count  303.000000  303.000000
mean     2.313531    0.544554
std      0.612277    0.498835
min      0.000000    0.000000
25\%      2.000000    0.000000
50\%      2.000000    1.000000
75\%      3.000000    1.000000
max      3.000000    1.000000
\end{Verbatim}
\end{tcolorbox}
        
    \hypertarget{heart-disease-frequency-according-to-sex}{%
\subsubsection{Heart Disease Frequency according to
Sex}\label{heart-disease-frequency-according-to-sex}}

    \begin{tcolorbox}[breakable, size=fbox, boxrule=1pt, pad at break*=1mm,colback=cellbackground, colframe=cellborder]
\prompt{In}{incolor}{12}{\boxspacing}
\begin{Verbatim}[commandchars=\\\{\}]
\PY{n}{df}\PY{o}{.}\PY{n}{sex}\PY{o}{.}\PY{n}{value\PYZus{}counts}\PY{p}{(}\PY{p}{)}
\end{Verbatim}
\end{tcolorbox}

            \begin{tcolorbox}[breakable, size=fbox, boxrule=.5pt, pad at break*=1mm, opacityfill=0]
\prompt{Out}{outcolor}{12}{\boxspacing}
\begin{Verbatim}[commandchars=\\\{\}]
1    207
0     96
Name: sex, dtype: int64
\end{Verbatim}
\end{tcolorbox}
        
    \begin{tcolorbox}[breakable, size=fbox, boxrule=1pt, pad at break*=1mm,colback=cellbackground, colframe=cellborder]
\prompt{In}{incolor}{13}{\boxspacing}
\begin{Verbatim}[commandchars=\\\{\}]
\PY{c+c1}{\PYZsh{} Compare taget column with sex column}
\PY{n}{pd}\PY{o}{.}\PY{n}{crosstab}\PY{p}{(}\PY{n}{df}\PY{o}{.}\PY{n}{target}\PY{p}{,} \PY{n}{df}\PY{o}{.}\PY{n}{sex}\PY{p}{)}
\end{Verbatim}
\end{tcolorbox}

            \begin{tcolorbox}[breakable, size=fbox, boxrule=.5pt, pad at break*=1mm, opacityfill=0]
\prompt{Out}{outcolor}{13}{\boxspacing}
\begin{Verbatim}[commandchars=\\\{\}]
sex      0    1
target
0       24  114
1       72   93
\end{Verbatim}
\end{tcolorbox}
        
    \begin{tcolorbox}[breakable, size=fbox, boxrule=1pt, pad at break*=1mm,colback=cellbackground, colframe=cellborder]
\prompt{In}{incolor}{14}{\boxspacing}
\begin{Verbatim}[commandchars=\\\{\}]
\PY{c+c1}{\PYZsh{} Create a plot of crosstab }
\PY{n}{pd}\PY{o}{.}\PY{n}{crosstab}\PY{p}{(}\PY{n}{df}\PY{o}{.}\PY{n}{target}\PY{p}{,} \PY{n}{df}\PY{o}{.}\PY{n}{sex}\PY{p}{)}\PY{o}{.}\PY{n}{plot}\PY{p}{(}\PY{n}{kind}\PY{o}{=}\PY{l+s+s2}{\PYZdq{}}\PY{l+s+s2}{bar}\PY{l+s+s2}{\PYZdq{}}\PY{p}{,}
                                   \PY{n}{figsize}\PY{o}{=}\PY{p}{(}\PY{l+m+mi}{10}\PY{p}{,}\PY{l+m+mi}{6}\PY{p}{)}\PY{p}{,}
                                   \PY{n}{color}\PY{o}{=}\PY{p}{[}\PY{l+s+s2}{\PYZdq{}}\PY{l+s+s2}{salmon}\PY{l+s+s2}{\PYZdq{}}\PY{p}{,}\PY{l+s+s2}{\PYZdq{}}\PY{l+s+s2}{lightblue}\PY{l+s+s2}{\PYZdq{}}\PY{p}{]}\PY{p}{)}
\PY{n}{plt}\PY{o}{.}\PY{n}{title}\PY{p}{(}\PY{l+s+s2}{\PYZdq{}}\PY{l+s+s2}{Heart Disease Frequency for sex}\PY{l+s+s2}{\PYZdq{}}\PY{p}{)}
\PY{n}{plt}\PY{o}{.}\PY{n}{xlabel}\PY{p}{(}\PY{l+s+s2}{\PYZdq{}}\PY{l+s+s2}{0 = No heart disease, 1 = Heart disease}\PY{l+s+s2}{\PYZdq{}}\PY{p}{)}
\PY{n}{plt}\PY{o}{.}\PY{n}{ylabel}\PY{p}{(}\PY{l+s+s2}{\PYZdq{}}\PY{l+s+s2}{Amount}\PY{l+s+s2}{\PYZdq{}}\PY{p}{)}
\PY{n}{plt}\PY{o}{.}\PY{n}{legend}\PY{p}{(}\PY{p}{[}\PY{l+s+s2}{\PYZdq{}}\PY{l+s+s2}{female}\PY{l+s+s2}{\PYZdq{}}\PY{p}{,} \PY{l+s+s2}{\PYZdq{}}\PY{l+s+s2}{male}\PY{l+s+s2}{\PYZdq{}}\PY{p}{]}\PY{p}{)}
\PY{n}{plt}\PY{o}{.}\PY{n}{xticks}\PY{p}{(}\PY{n}{rotation}\PY{o}{=}\PY{l+m+mi}{0}\PY{p}{)}\PY{p}{;}
            
\end{Verbatim}
\end{tcolorbox}

    \begin{center}
    \adjustimage{max size={0.9\linewidth}{0.9\paperheight}}{output_22_0.png}
    \end{center}
    { \hspace*{\fill} \\}
    
    \begin{tcolorbox}[breakable, size=fbox, boxrule=1pt, pad at break*=1mm,colback=cellbackground, colframe=cellborder]
\prompt{In}{incolor}{15}{\boxspacing}
\begin{Verbatim}[commandchars=\\\{\}]
\PY{n}{df}\PY{o}{.}\PY{n}{head}\PY{p}{(}\PY{p}{)}
\end{Verbatim}
\end{tcolorbox}

            \begin{tcolorbox}[breakable, size=fbox, boxrule=.5pt, pad at break*=1mm, opacityfill=0]
\prompt{Out}{outcolor}{15}{\boxspacing}
\begin{Verbatim}[commandchars=\\\{\}]
   age  sex  cp  trestbps  chol  fbs  restecg  thalach  exang  oldpeak  slope  \textbackslash{}
0   63    1   3       145   233    1        0      150      0      2.3      0
1   37    1   2       130   250    0        1      187      0      3.5      0
2   41    0   1       130   204    0        0      172      0      1.4      2
3   56    1   1       120   236    0        1      178      0      0.8      2
4   57    0   0       120   354    0        1      163      1      0.6      2

   ca  thal  target
0   0     1       1
1   0     2       1
2   0     2       1
3   0     2       1
4   0     2       1
\end{Verbatim}
\end{tcolorbox}
        
    \begin{tcolorbox}[breakable, size=fbox, boxrule=1pt, pad at break*=1mm,colback=cellbackground, colframe=cellborder]
\prompt{In}{incolor}{16}{\boxspacing}
\begin{Verbatim}[commandchars=\\\{\}]
\PY{n}{df}\PY{p}{[}\PY{l+s+s2}{\PYZdq{}}\PY{l+s+s2}{thalach}\PY{l+s+s2}{\PYZdq{}}\PY{p}{]}\PY{o}{.}\PY{n}{value\PYZus{}counts}\PY{p}{(}\PY{p}{)}
\end{Verbatim}
\end{tcolorbox}

            \begin{tcolorbox}[breakable, size=fbox, boxrule=.5pt, pad at break*=1mm, opacityfill=0]
\prompt{Out}{outcolor}{16}{\boxspacing}
\begin{Verbatim}[commandchars=\\\{\}]
162    11
160     9
163     9
173     8
152     8
       ..
129     1
128     1
127     1
124     1
71      1
Name: thalach, Length: 91, dtype: int64
\end{Verbatim}
\end{tcolorbox}
        
    \hypertarget{age-vs-max-heart-rate-for-heart-disease}{%
\subsubsection{Age vs Max Heart Rate for Heart
Disease}\label{age-vs-max-heart-rate-for-heart-disease}}

    \begin{tcolorbox}[breakable, size=fbox, boxrule=1pt, pad at break*=1mm,colback=cellbackground, colframe=cellborder]
\prompt{In}{incolor}{17}{\boxspacing}
\begin{Verbatim}[commandchars=\\\{\}]
\PY{c+c1}{\PYZsh{} Create another figure}
\PY{n}{plt}\PY{o}{.}\PY{n}{figure}\PY{p}{(}\PY{n}{figsize}\PY{o}{=}\PY{p}{(}\PY{l+m+mi}{10}\PY{p}{,}\PY{l+m+mi}{6}\PY{p}{)}\PY{p}{)}

\PY{c+c1}{\PYZsh{}Scatter with positive examples}
\PY{n}{plt}\PY{o}{.}\PY{n}{scatter}\PY{p}{(}\PY{n}{df}\PY{o}{.}\PY{n}{age}\PY{p}{[}\PY{n}{df}\PY{o}{.}\PY{n}{target}\PY{o}{==}\PY{l+m+mi}{1}\PY{p}{]}\PY{p}{,}
           \PY{n}{df}\PY{o}{.}\PY{n}{thalach}\PY{p}{[}\PY{n}{df}\PY{o}{.}\PY{n}{target}\PY{o}{==}\PY{l+m+mi}{1}\PY{p}{]}\PY{p}{,}
           \PY{n}{c}\PY{o}{=}\PY{l+s+s2}{\PYZdq{}}\PY{l+s+s2}{salmon}\PY{l+s+s2}{\PYZdq{}}\PY{p}{)}

\PY{c+c1}{\PYZsh{} Scatter with negative examples}
\PY{n}{plt}\PY{o}{.}\PY{n}{scatter}\PY{p}{(}\PY{n}{df}\PY{o}{.}\PY{n}{age}\PY{p}{[}\PY{n}{df}\PY{o}{.}\PY{n}{target}\PY{o}{==}\PY{l+m+mi}{0}\PY{p}{]}\PY{p}{,}
           \PY{n}{df}\PY{o}{.}\PY{n}{thalach}\PY{p}{[}\PY{n}{df}\PY{o}{.}\PY{n}{target}\PY{o}{==}\PY{l+m+mi}{0}\PY{p}{]}\PY{p}{,}
           \PY{n}{c}\PY{o}{=}\PY{l+s+s2}{\PYZdq{}}\PY{l+s+s2}{lightblue}\PY{l+s+s2}{\PYZdq{}}\PY{p}{)}\PY{p}{;}

\PY{c+c1}{\PYZsh{} Add some helpfull info}
\PY{n}{plt}\PY{o}{.}\PY{n}{title}\PY{p}{(}\PY{l+s+s2}{\PYZdq{}}\PY{l+s+s2}{heart disease in function od Age and MAx Heart Rate}\PY{l+s+s2}{\PYZdq{}}\PY{p}{)}
\PY{n}{plt}\PY{o}{.}\PY{n}{xlabel}\PY{p}{(}\PY{l+s+s2}{\PYZdq{}}\PY{l+s+s2}{Age}\PY{l+s+s2}{\PYZdq{}}\PY{p}{)}
\PY{n}{plt}\PY{o}{.}\PY{n}{ylabel}\PY{p}{(}\PY{l+s+s2}{\PYZdq{}}\PY{l+s+s2}{Ma Heart rate}\PY{l+s+s2}{\PYZdq{}}\PY{p}{)}
\PY{n}{plt}\PY{o}{.}\PY{n}{legend}\PY{p}{(}\PY{p}{[}\PY{l+s+s2}{\PYZdq{}}\PY{l+s+s2}{disease}\PY{l+s+s2}{\PYZdq{}}\PY{p}{,}\PY{l+s+s2}{\PYZdq{}}\PY{l+s+s2}{no disease}\PY{l+s+s2}{\PYZdq{}}\PY{p}{]}\PY{p}{)}\PY{p}{;}
\end{Verbatim}
\end{tcolorbox}

    \begin{center}
    \adjustimage{max size={0.9\linewidth}{0.9\paperheight}}{output_26_0.png}
    \end{center}
    { \hspace*{\fill} \\}
    
    \begin{tcolorbox}[breakable, size=fbox, boxrule=1pt, pad at break*=1mm,colback=cellbackground, colframe=cellborder]
\prompt{In}{incolor}{18}{\boxspacing}
\begin{Verbatim}[commandchars=\\\{\}]
\PY{c+c1}{\PYZsh{} Check the distribution of the age column with a histogrm }
\PY{n}{df}\PY{o}{.}\PY{n}{age}\PY{o}{.}\PY{n}{plot}\PY{o}{.}\PY{n}{hist}\PY{p}{(}\PY{p}{)}\PY{p}{;}
\end{Verbatim}
\end{tcolorbox}

    \begin{center}
    \adjustimage{max size={0.9\linewidth}{0.9\paperheight}}{output_27_0.png}
    \end{center}
    { \hspace*{\fill} \\}
    
    \hypertarget{heart-disease-frequency-per-chest-pain-type}{%
\subsubsection{Heart Disease frequency per Chest Pain
Type}\label{heart-disease-frequency-per-chest-pain-type}}

    \begin{tcolorbox}[breakable, size=fbox, boxrule=1pt, pad at break*=1mm,colback=cellbackground, colframe=cellborder]
\prompt{In}{incolor}{19}{\boxspacing}
\begin{Verbatim}[commandchars=\\\{\}]
\PY{n}{pd}\PY{o}{.}\PY{n}{crosstab}\PY{p}{(}\PY{n}{df}\PY{o}{.}\PY{n}{cp}\PY{p}{,} \PY{n}{df}\PY{o}{.}\PY{n}{target}\PY{p}{)}
\end{Verbatim}
\end{tcolorbox}

            \begin{tcolorbox}[breakable, size=fbox, boxrule=.5pt, pad at break*=1mm, opacityfill=0]
\prompt{Out}{outcolor}{19}{\boxspacing}
\begin{Verbatim}[commandchars=\\\{\}]
target    0   1
cp
0       104  39
1         9  41
2        18  69
3         7  16
\end{Verbatim}
\end{tcolorbox}
        
    \begin{tcolorbox}[breakable, size=fbox, boxrule=1pt, pad at break*=1mm,colback=cellbackground, colframe=cellborder]
\prompt{In}{incolor}{20}{\boxspacing}
\begin{Verbatim}[commandchars=\\\{\}]
\PY{c+c1}{\PYZsh{} Make the crosstab more visual }
\PY{n}{pd}\PY{o}{.}\PY{n}{crosstab}\PY{p}{(}\PY{n}{df}\PY{o}{.}\PY{n}{cp}\PY{p}{,} \PY{n}{df}\PY{o}{.}\PY{n}{target}\PY{p}{)}\PY{o}{.}\PY{n}{plot}\PY{p}{(}\PY{n}{kind}\PY{o}{=}\PY{l+s+s2}{\PYZdq{}}\PY{l+s+s2}{bar}\PY{l+s+s2}{\PYZdq{}}\PY{p}{,}
                                  \PY{n}{figsize}\PY{o}{=}\PY{p}{(}\PY{l+m+mi}{10}\PY{p}{,}\PY{l+m+mi}{6}\PY{p}{)}\PY{p}{,}
                                \PY{n}{color}\PY{o}{=}\PY{p}{[}\PY{l+s+s2}{\PYZdq{}}\PY{l+s+s2}{lightblue}\PY{l+s+s2}{\PYZdq{}}\PY{p}{,} \PY{l+s+s2}{\PYZdq{}}\PY{l+s+s2}{salmon}\PY{l+s+s2}{\PYZdq{}}\PY{p}{]}\PY{p}{)}

\PY{c+c1}{\PYZsh{} Add some communication}
\PY{n}{plt}\PY{o}{.}\PY{n}{title}\PY{p}{(}\PY{l+s+s2}{\PYZdq{}}\PY{l+s+s2}{Heart Disease Frequency Per Chest Pain Type}\PY{l+s+s2}{\PYZdq{}}\PY{p}{)}
\PY{n}{plt}\PY{o}{.}\PY{n}{xlabel}\PY{p}{(}\PY{l+s+s2}{\PYZdq{}}\PY{l+s+s2}{chest pain Type}\PY{l+s+s2}{\PYZdq{}}\PY{p}{)}
\PY{n}{plt}\PY{o}{.}\PY{n}{ylabel}\PY{p}{(}\PY{l+s+s2}{\PYZdq{}}\PY{l+s+s2}{Amount}\PY{l+s+s2}{\PYZdq{}}\PY{p}{)}
\PY{n}{plt}\PY{o}{.}\PY{n}{legend}\PY{p}{(}\PY{p}{[}\PY{l+s+s2}{\PYZdq{}}\PY{l+s+s2}{no disease}\PY{l+s+s2}{\PYZdq{}}\PY{p}{,} \PY{l+s+s2}{\PYZdq{}}\PY{l+s+s2}{Disease}\PY{l+s+s2}{\PYZdq{}}\PY{p}{]}\PY{p}{)}
\PY{n}{plt}\PY{o}{.}\PY{n}{xticks}\PY{p}{(}\PY{n}{rotation}\PY{o}{=}\PY{l+m+mi}{0}\PY{p}{)}\PY{p}{;}
\end{Verbatim}
\end{tcolorbox}

    \begin{center}
    \adjustimage{max size={0.9\linewidth}{0.9\paperheight}}{output_30_0.png}
    \end{center}
    { \hspace*{\fill} \\}
    
    \begin{tcolorbox}[breakable, size=fbox, boxrule=1pt, pad at break*=1mm,colback=cellbackground, colframe=cellborder]
\prompt{In}{incolor}{21}{\boxspacing}
\begin{Verbatim}[commandchars=\\\{\}]
\PY{n}{df}\PY{o}{.}\PY{n}{head}\PY{p}{(}\PY{p}{)}
\end{Verbatim}
\end{tcolorbox}

            \begin{tcolorbox}[breakable, size=fbox, boxrule=.5pt, pad at break*=1mm, opacityfill=0]
\prompt{Out}{outcolor}{21}{\boxspacing}
\begin{Verbatim}[commandchars=\\\{\}]
   age  sex  cp  trestbps  chol  fbs  restecg  thalach  exang  oldpeak  slope  \textbackslash{}
0   63    1   3       145   233    1        0      150      0      2.3      0
1   37    1   2       130   250    0        1      187      0      3.5      0
2   41    0   1       130   204    0        0      172      0      1.4      2
3   56    1   1       120   236    0        1      178      0      0.8      2
4   57    0   0       120   354    0        1      163      1      0.6      2

   ca  thal  target
0   0     1       1
1   0     2       1
2   0     2       1
3   0     2       1
4   0     2       1
\end{Verbatim}
\end{tcolorbox}
        
    \begin{tcolorbox}[breakable, size=fbox, boxrule=1pt, pad at break*=1mm,colback=cellbackground, colframe=cellborder]
\prompt{In}{incolor}{22}{\boxspacing}
\begin{Verbatim}[commandchars=\\\{\}]
\PY{c+c1}{\PYZsh{} Make a correlation matrix }
\PY{n}{df}\PY{o}{.}\PY{n}{corr}\PY{p}{(}\PY{p}{)}
\end{Verbatim}
\end{tcolorbox}

            \begin{tcolorbox}[breakable, size=fbox, boxrule=.5pt, pad at break*=1mm, opacityfill=0]
\prompt{Out}{outcolor}{22}{\boxspacing}
\begin{Verbatim}[commandchars=\\\{\}]
               age       sex        cp  trestbps      chol       fbs  \textbackslash{}
age       1.000000 -0.098447 -0.068653  0.279351  0.213678  0.121308
sex      -0.098447  1.000000 -0.049353 -0.056769 -0.197912  0.045032
cp       -0.068653 -0.049353  1.000000  0.047608 -0.076904  0.094444
trestbps  0.279351 -0.056769  0.047608  1.000000  0.123174  0.177531
chol      0.213678 -0.197912 -0.076904  0.123174  1.000000  0.013294
fbs       0.121308  0.045032  0.094444  0.177531  0.013294  1.000000
restecg  -0.116211 -0.058196  0.044421 -0.114103 -0.151040 -0.084189
thalach  -0.398522 -0.044020  0.295762 -0.046698 -0.009940 -0.008567
exang     0.096801  0.141664 -0.394280  0.067616  0.067023  0.025665
oldpeak   0.210013  0.096093 -0.149230  0.193216  0.053952  0.005747
slope    -0.168814 -0.030711  0.119717 -0.121475 -0.004038 -0.059894
ca        0.276326  0.118261 -0.181053  0.101389  0.070511  0.137979
thal      0.068001  0.210041 -0.161736  0.062210  0.098803 -0.032019
target   -0.225439 -0.280937  0.433798 -0.144931 -0.085239 -0.028046

           restecg   thalach     exang   oldpeak     slope        ca  \textbackslash{}
age      -0.116211 -0.398522  0.096801  0.210013 -0.168814  0.276326
sex      -0.058196 -0.044020  0.141664  0.096093 -0.030711  0.118261
cp        0.044421  0.295762 -0.394280 -0.149230  0.119717 -0.181053
trestbps -0.114103 -0.046698  0.067616  0.193216 -0.121475  0.101389
chol     -0.151040 -0.009940  0.067023  0.053952 -0.004038  0.070511
fbs      -0.084189 -0.008567  0.025665  0.005747 -0.059894  0.137979
restecg   1.000000  0.044123 -0.070733 -0.058770  0.093045 -0.072042
thalach   0.044123  1.000000 -0.378812 -0.344187  0.386784 -0.213177
exang    -0.070733 -0.378812  1.000000  0.288223 -0.257748  0.115739
oldpeak  -0.058770 -0.344187  0.288223  1.000000 -0.577537  0.222682
slope     0.093045  0.386784 -0.257748 -0.577537  1.000000 -0.080155
ca       -0.072042 -0.213177  0.115739  0.222682 -0.080155  1.000000
thal     -0.011981 -0.096439  0.206754  0.210244 -0.104764  0.151832
target    0.137230  0.421741 -0.436757 -0.430696  0.345877 -0.391724

              thal    target
age       0.068001 -0.225439
sex       0.210041 -0.280937
cp       -0.161736  0.433798
trestbps  0.062210 -0.144931
chol      0.098803 -0.085239
fbs      -0.032019 -0.028046
restecg  -0.011981  0.137230
thalach  -0.096439  0.421741
exang     0.206754 -0.436757
oldpeak   0.210244 -0.430696
slope    -0.104764  0.345877
ca        0.151832 -0.391724
thal      1.000000 -0.344029
target   -0.344029  1.000000
\end{Verbatim}
\end{tcolorbox}
        
    \begin{tcolorbox}[breakable, size=fbox, boxrule=1pt, pad at break*=1mm,colback=cellbackground, colframe=cellborder]
\prompt{In}{incolor}{23}{\boxspacing}
\begin{Verbatim}[commandchars=\\\{\}]
\PY{c+c1}{\PYZsh{} lets make our correlation matrix  little prettier }
\PY{n}{corr\PYZus{}matrix} \PY{o}{=} \PY{n}{df}\PY{o}{.}\PY{n}{corr}\PY{p}{(}\PY{p}{)}
\PY{n}{fig}\PY{p}{,} \PY{n}{ax} \PY{o}{=} \PY{n}{plt}\PY{o}{.}\PY{n}{subplots}\PY{p}{(}\PY{n}{figsize}\PY{o}{=} \PY{p}{(}\PY{l+m+mi}{20}\PY{p}{,} \PY{l+m+mi}{15}\PY{p}{)}\PY{p}{)}
\PY{n}{ax} \PY{o}{=} \PY{n}{sns}\PY{o}{.}\PY{n}{heatmap}\PY{p}{(}\PY{n}{corr\PYZus{}matrix}\PY{p}{,} 
                \PY{n}{annot}\PY{o}{=}\PY{k+kc}{True}\PY{p}{,}  
                \PY{n}{linewidths}\PY{o}{=} \PY{l+m+mf}{0.5}\PY{p}{,} 
                \PY{n}{fmt} \PY{o}{=}\PY{l+s+s1}{\PYZsq{}}\PY{l+s+s1}{.2f}\PY{l+s+s1}{\PYZsq{}}\PY{p}{,} 
                \PY{n}{cmap} \PY{o}{=} \PY{l+s+s1}{\PYZsq{}}\PY{l+s+s1}{YlGnBu}\PY{l+s+s1}{\PYZsq{}}\PY{p}{)}\PY{p}{;}
\end{Verbatim}
\end{tcolorbox}

    \begin{center}
    \adjustimage{max size={0.9\linewidth}{0.9\paperheight}}{output_33_0.png}
    \end{center}
    { \hspace*{\fill} \\}
    
    \begin{tcolorbox}[breakable, size=fbox, boxrule=1pt, pad at break*=1mm,colback=cellbackground, colframe=cellborder]
\prompt{In}{incolor}{24}{\boxspacing}
\begin{Verbatim}[commandchars=\\\{\}]
\PY{c+c1}{\PYZsh{} double check correlations fro Exang  }
\PY{n}{pd}\PY{o}{.}\PY{n}{crosstab}\PY{p}{(}\PY{n}{df}\PY{o}{.}\PY{n}{exang}\PY{p}{,} \PY{n}{df}\PY{o}{.}\PY{n}{target}\PY{p}{)}\PY{o}{.}\PY{n}{plot}\PY{p}{(}\PY{n}{kind}\PY{o}{=}\PY{l+s+s2}{\PYZdq{}}\PY{l+s+s2}{bar}\PY{l+s+s2}{\PYZdq{}}\PY{p}{,}
                                  \PY{n}{figsize}\PY{o}{=}\PY{p}{(}\PY{l+m+mi}{10}\PY{p}{,}\PY{l+m+mi}{6}\PY{p}{)}\PY{p}{,}
                                \PY{n}{color}\PY{o}{=}\PY{p}{[}\PY{l+s+s2}{\PYZdq{}}\PY{l+s+s2}{lightblue}\PY{l+s+s2}{\PYZdq{}}\PY{p}{,} \PY{l+s+s2}{\PYZdq{}}\PY{l+s+s2}{salmon}\PY{l+s+s2}{\PYZdq{}}\PY{p}{]}\PY{p}{)}

\PY{c+c1}{\PYZsh{} Add some communication}
\PY{n}{plt}\PY{o}{.}\PY{n}{title}\PY{p}{(}\PY{l+s+s2}{\PYZdq{}}\PY{l+s+s2}{Heart Disease Frequency Per Exercice induce chest pain}\PY{l+s+s2}{\PYZdq{}}\PY{p}{)}
\PY{n}{plt}\PY{o}{.}\PY{n}{xlabel}\PY{p}{(}\PY{l+s+s2}{\PYZdq{}}\PY{l+s+s2}{Exang}\PY{l+s+s2}{\PYZdq{}}\PY{p}{)}
\PY{n}{plt}\PY{o}{.}\PY{n}{ylabel}\PY{p}{(}\PY{l+s+s2}{\PYZdq{}}\PY{l+s+s2}{Amount}\PY{l+s+s2}{\PYZdq{}}\PY{p}{)}
\PY{n}{plt}\PY{o}{.}\PY{n}{legend}\PY{p}{(}\PY{p}{[}\PY{l+s+s2}{\PYZdq{}}\PY{l+s+s2}{no disease}\PY{l+s+s2}{\PYZdq{}}\PY{p}{,} \PY{l+s+s2}{\PYZdq{}}\PY{l+s+s2}{Disease}\PY{l+s+s2}{\PYZdq{}}\PY{p}{]}\PY{p}{)}
\PY{n}{plt}\PY{o}{.}\PY{n}{xticks}\PY{p}{(}\PY{n}{rotation}\PY{o}{=}\PY{l+m+mi}{0}\PY{p}{)}\PY{p}{;}
\end{Verbatim}
\end{tcolorbox}

    \begin{center}
    \adjustimage{max size={0.9\linewidth}{0.9\paperheight}}{output_34_0.png}
    \end{center}
    { \hspace*{\fill} \\}
    
    \hypertarget{modelling}{%
\subsection{5. Modelling}\label{modelling}}

    \begin{tcolorbox}[breakable, size=fbox, boxrule=1pt, pad at break*=1mm,colback=cellbackground, colframe=cellborder]
\prompt{In}{incolor}{25}{\boxspacing}
\begin{Verbatim}[commandchars=\\\{\}]
\PY{n}{df}\PY{o}{.}\PY{n}{head}\PY{p}{(}\PY{p}{)}
\end{Verbatim}
\end{tcolorbox}

            \begin{tcolorbox}[breakable, size=fbox, boxrule=.5pt, pad at break*=1mm, opacityfill=0]
\prompt{Out}{outcolor}{25}{\boxspacing}
\begin{Verbatim}[commandchars=\\\{\}]
   age  sex  cp  trestbps  chol  fbs  restecg  thalach  exang  oldpeak  slope  \textbackslash{}
0   63    1   3       145   233    1        0      150      0      2.3      0
1   37    1   2       130   250    0        1      187      0      3.5      0
2   41    0   1       130   204    0        0      172      0      1.4      2
3   56    1   1       120   236    0        1      178      0      0.8      2
4   57    0   0       120   354    0        1      163      1      0.6      2

   ca  thal  target
0   0     1       1
1   0     2       1
2   0     2       1
3   0     2       1
4   0     2       1
\end{Verbatim}
\end{tcolorbox}
        
    \begin{tcolorbox}[breakable, size=fbox, boxrule=1pt, pad at break*=1mm,colback=cellbackground, colframe=cellborder]
\prompt{In}{incolor}{26}{\boxspacing}
\begin{Verbatim}[commandchars=\\\{\}]
\PY{c+c1}{\PYZsh{} Split data into X and y }
\PY{n}{X} \PY{o}{=}  \PY{n}{df}\PY{o}{.}\PY{n}{drop}\PY{p}{(}\PY{l+s+s2}{\PYZdq{}}\PY{l+s+s2}{target}\PY{l+s+s2}{\PYZdq{}}\PY{p}{,} \PY{n}{axis}\PY{o}{=}\PY{l+m+mi}{1}\PY{p}{)}
\PY{n}{y} \PY{o}{=} \PY{n}{df}\PY{p}{[}\PY{l+s+s2}{\PYZdq{}}\PY{l+s+s2}{target}\PY{l+s+s2}{\PYZdq{}}\PY{p}{]}
\end{Verbatim}
\end{tcolorbox}

    \begin{tcolorbox}[breakable, size=fbox, boxrule=1pt, pad at break*=1mm,colback=cellbackground, colframe=cellborder]
\prompt{In}{incolor}{27}{\boxspacing}
\begin{Verbatim}[commandchars=\\\{\}]
\PY{n}{X}
\end{Verbatim}
\end{tcolorbox}

            \begin{tcolorbox}[breakable, size=fbox, boxrule=.5pt, pad at break*=1mm, opacityfill=0]
\prompt{Out}{outcolor}{27}{\boxspacing}
\begin{Verbatim}[commandchars=\\\{\}]
     age  sex  cp  trestbps  chol  fbs  restecg  thalach  exang  oldpeak  \textbackslash{}
0     63    1   3       145   233    1        0      150      0      2.3
1     37    1   2       130   250    0        1      187      0      3.5
2     41    0   1       130   204    0        0      172      0      1.4
3     56    1   1       120   236    0        1      178      0      0.8
4     57    0   0       120   354    0        1      163      1      0.6
..   {\ldots}  {\ldots}  ..       {\ldots}   {\ldots}  {\ldots}      {\ldots}      {\ldots}    {\ldots}      {\ldots}
298   57    0   0       140   241    0        1      123      1      0.2
299   45    1   3       110   264    0        1      132      0      1.2
300   68    1   0       144   193    1        1      141      0      3.4
301   57    1   0       130   131    0        1      115      1      1.2
302   57    0   1       130   236    0        0      174      0      0.0

     slope  ca  thal
0        0   0     1
1        0   0     2
2        2   0     2
3        2   0     2
4        2   0     2
..     {\ldots}  ..   {\ldots}
298      1   0     3
299      1   0     3
300      1   2     3
301      1   1     3
302      1   1     2

[303 rows x 13 columns]
\end{Verbatim}
\end{tcolorbox}
        
    \begin{tcolorbox}[breakable, size=fbox, boxrule=1pt, pad at break*=1mm,colback=cellbackground, colframe=cellborder]
\prompt{In}{incolor}{28}{\boxspacing}
\begin{Verbatim}[commandchars=\\\{\}]
\PY{n}{y}
\end{Verbatim}
\end{tcolorbox}

            \begin{tcolorbox}[breakable, size=fbox, boxrule=.5pt, pad at break*=1mm, opacityfill=0]
\prompt{Out}{outcolor}{28}{\boxspacing}
\begin{Verbatim}[commandchars=\\\{\}]
0      1
1      1
2      1
3      1
4      1
      ..
298    0
299    0
300    0
301    0
302    0
Name: target, Length: 303, dtype: int64
\end{Verbatim}
\end{tcolorbox}
        
    \begin{tcolorbox}[breakable, size=fbox, boxrule=1pt, pad at break*=1mm,colback=cellbackground, colframe=cellborder]
\prompt{In}{incolor}{29}{\boxspacing}
\begin{Verbatim}[commandchars=\\\{\}]
\PY{c+c1}{\PYZsh{} Split data into train and test sets}
\PY{n}{np}\PY{o}{.}\PY{n}{random}\PY{o}{.}\PY{n}{seed}\PY{p}{(}\PY{l+m+mi}{42}\PY{p}{)}

\PY{c+c1}{\PYZsh{} Split into train and test set}
\PY{n}{X\PYZus{}train}\PY{p}{,} \PY{n}{X\PYZus{}test}\PY{p}{,} \PY{n}{y\PYZus{}train}\PY{p}{,} \PY{n}{y\PYZus{}test} \PY{o}{=} \PY{n}{train\PYZus{}test\PYZus{}split}\PY{p}{(}\PY{n}{X}\PY{p}{,}
                                                   \PY{n}{y}\PY{p}{,}
                                                   \PY{n}{test\PYZus{}size} \PY{o}{=} \PY{l+m+mf}{0.2}\PY{p}{)}
\end{Verbatim}
\end{tcolorbox}

    \begin{tcolorbox}[breakable, size=fbox, boxrule=1pt, pad at break*=1mm,colback=cellbackground, colframe=cellborder]
\prompt{In}{incolor}{30}{\boxspacing}
\begin{Verbatim}[commandchars=\\\{\}]
\PY{n}{y\PYZus{}train}\PY{p}{,} \PY{n+nb}{len}\PY{p}{(}\PY{n}{y\PYZus{}train}\PY{p}{)}
\end{Verbatim}
\end{tcolorbox}

            \begin{tcolorbox}[breakable, size=fbox, boxrule=.5pt, pad at break*=1mm, opacityfill=0]
\prompt{Out}{outcolor}{30}{\boxspacing}
\begin{Verbatim}[commandchars=\\\{\}]
(132    1
 202    0
 196    0
 75     1
 176    0
       ..
 188    0
 71     1
 106    1
 270    0
 102    1
 Name: target, Length: 242, dtype: int64,
 242)
\end{Verbatim}
\end{tcolorbox}
        
    Now we've got our data split into train and test set its time to build
our machine learning model. We'll train it (find the patterns) on the
training set. We'll test it (use the patterns) on the teest set.

we are going to try 3 differnet machine learning models : 1. Logistic
Regression 2. K-nearest neighbour calssifier 3. Random forest classifier

    \begin{tcolorbox}[breakable, size=fbox, boxrule=1pt, pad at break*=1mm,colback=cellbackground, colframe=cellborder]
\prompt{In}{incolor}{32}{\boxspacing}
\begin{Verbatim}[commandchars=\\\{\}]
\PY{c+c1}{\PYZsh{} looking for the right model }
\PY{n}{models} \PY{o}{=} \PY{p}{\PYZob{}}\PY{l+s+s1}{\PYZsq{}}\PY{l+s+s1}{Logistic Regression}\PY{l+s+s1}{\PYZsq{}}\PY{p}{:} \PY{n}{LogisticRegression}\PY{p}{(}\PY{p}{)}\PY{p}{,}
         \PY{l+s+s1}{\PYZsq{}}\PY{l+s+s1}{KNN}\PY{l+s+s1}{\PYZsq{}}\PY{p}{:} \PY{n}{KNeighborsClassifier}\PY{p}{(}\PY{p}{)}\PY{p}{,}
         \PY{l+s+s1}{\PYZsq{}}\PY{l+s+s1}{Random Forest}\PY{l+s+s1}{\PYZsq{}}\PY{p}{:} \PY{n}{RandomForestClassifier}\PY{p}{(}\PY{p}{)}\PY{p}{\PYZcb{}}

\PY{c+c1}{\PYZsh{}create a function to fit andscore models }

\PY{k}{def} \PY{n+nf}{fit\PYZus{}and\PYZus{}score}\PY{p}{(}\PY{n}{models}\PY{p}{,} \PY{n}{X\PYZus{}train}\PY{p}{,} \PY{n}{X\PYZus{}test}\PY{p}{,} \PY{n}{y\PYZus{}train}\PY{p}{,} \PY{n}{y\PYZus{}test}\PY{p}{)}\PY{p}{:}
    \PY{l+s+sd}{\PYZdq{}\PYZdq{}\PYZdq{}}
\PY{l+s+sd}{    Fits and evalueats given machine learning models.}
\PY{l+s+sd}{    models: a dict of different SCikit\PYZhy{}learn machine learning models }
\PY{l+s+sd}{    X\PYZus{}train : trainig data (no labesl )}
\PY{l+s+sd}{    X\PYZus{}test: testing data (no labels)}
\PY{l+s+sd}{    y\PYZus{}train: training labesl }
\PY{l+s+sd}{    y\PYZus{}test: test labels }
\PY{l+s+sd}{    \PYZdq{}\PYZdq{}\PYZdq{}}
    \PY{c+c1}{\PYZsh{}set random seed }
    \PY{n}{np}\PY{o}{.}\PY{n}{random}\PY{o}{.}\PY{n}{seed}\PY{p}{(}\PY{l+m+mi}{42}\PY{p}{)}
    \PY{c+c1}{\PYZsh{}Make a dictionary  to keep models scorees }
    \PY{n}{model\PYZus{}scores}\PY{o}{=} \PY{p}{\PYZob{}}\PY{p}{\PYZcb{}}
    \PY{c+c1}{\PYZsh{}loop throug models }
    \PY{k}{for} \PY{n}{name}\PY{p}{,} \PY{n}{model} \PY{o+ow}{in} \PY{n}{models}\PY{o}{.}\PY{n}{items}\PY{p}{(}\PY{p}{)}\PY{p}{:}
        \PY{c+c1}{\PYZsh{}fit the modelto data }
        \PY{n}{model}\PY{o}{.}\PY{n}{fit}\PY{p}{(}\PY{n}{X\PYZus{}train}\PY{p}{,} \PY{n}{y\PYZus{}train}\PY{p}{)}
        \PY{c+c1}{\PYZsh{}Evaluate the modeland append its score to model\PYZus{}scores }
        \PY{n}{model\PYZus{}scores}\PY{p}{[}\PY{n}{name}\PY{p}{]} \PY{o}{=} \PY{n}{model}\PY{o}{.}\PY{n}{score}\PY{p}{(}\PY{n}{X\PYZus{}test}\PY{p}{,} \PY{n}{y\PYZus{}test}\PY{p}{)}
    \PY{k}{return} \PY{n}{model\PYZus{}scores}
\end{Verbatim}
\end{tcolorbox}

    \begin{tcolorbox}[breakable, size=fbox, boxrule=1pt, pad at break*=1mm,colback=cellbackground, colframe=cellborder]
\prompt{In}{incolor}{34}{\boxspacing}
\begin{Verbatim}[commandchars=\\\{\}]
\PY{n}{models\PYZus{}score} \PY{o}{=} \PY{n}{fit\PYZus{}and\PYZus{}score}\PY{p}{(}\PY{n}{models}\PY{o}{=} \PY{n}{models}\PY{p}{,} 
                            \PY{n}{X\PYZus{}train} \PY{o}{=} \PY{n}{X\PYZus{}train}\PY{p}{,}
                            \PY{n}{X\PYZus{}test} \PY{o}{=} \PY{n}{X\PYZus{}test}\PY{p}{,} 
                            \PY{n}{y\PYZus{}train} \PY{o}{=} \PY{n}{y\PYZus{}train}\PY{p}{,} 
                            \PY{n}{y\PYZus{}test} \PY{o}{=} \PY{n}{y\PYZus{}test}\PY{p}{)}
\PY{n}{models\PYZus{}score}
\end{Verbatim}
\end{tcolorbox}

    \begin{Verbatim}[commandchars=\\\{\}]
C:\textbackslash{}Users\textbackslash{}Owner\textbackslash{}Desktop\textbackslash{}Machine\_learning\textbackslash{}project\_mileston1\textbackslash{}env\textbackslash{}lib\textbackslash{}site-
packages\textbackslash{}sklearn\textbackslash{}linear\_model\textbackslash{}\_logistic.py:938: ConvergenceWarning: lbfgs failed
to converge (status=1):
STOP: TOTAL NO. of ITERATIONS REACHED LIMIT.

Increase the number of iterations (max\_iter) or scale the data as shown in:
    https://scikit-learn.org/stable/modules/preprocessing.html
Please also refer to the documentation for alternative solver options:
    https://scikit-learn.org/stable/modules/linear\_model.html\#logistic-
regression
  n\_iter\_i = \_check\_optimize\_result(
    \end{Verbatim}

            \begin{tcolorbox}[breakable, size=fbox, boxrule=.5pt, pad at break*=1mm, opacityfill=0]
\prompt{Out}{outcolor}{34}{\boxspacing}
\begin{Verbatim}[commandchars=\\\{\}]
\{'Logistic Regression': 0.8852459016393442,
 'KNN': 0.6885245901639344,
 'Random Forest': 0.8360655737704918\}
\end{Verbatim}
\end{tcolorbox}
        
    \hypertarget{model-comparision}{%
\subsubsection{Model comparision}\label{model-comparision}}

    \begin{tcolorbox}[breakable, size=fbox, boxrule=1pt, pad at break*=1mm,colback=cellbackground, colframe=cellborder]
\prompt{In}{incolor}{37}{\boxspacing}
\begin{Verbatim}[commandchars=\\\{\}]
\PY{n}{model\PYZus{}compare} \PY{o}{=} \PY{n}{pd}\PY{o}{.}\PY{n}{DataFrame}\PY{p}{(}\PY{n}{models\PYZus{}score}\PY{p}{,} \PY{n}{index}\PY{o}{=}\PY{p}{[}\PY{l+s+s1}{\PYZsq{}}\PY{l+s+s1}{accuracy}\PY{l+s+s1}{\PYZsq{}}\PY{p}{]}\PY{p}{)}
\PY{n}{model\PYZus{}compare}\PY{o}{.}\PY{n}{T}\PY{o}{.}\PY{n}{plot}\PY{o}{.}\PY{n}{bar}\PY{p}{(}\PY{p}{)}\PY{p}{;}
\end{Verbatim}
\end{tcolorbox}

    \begin{center}
    \adjustimage{max size={0.9\linewidth}{0.9\paperheight}}{output_46_0.png}
    \end{center}
    { \hspace*{\fill} \\}
    
    Now we've got a basseline model \ldots{} and we know a model's first
predictions and allway what we should we based our next steps of.

Let's look at following: * Hypperparameter tuning * feature importance *
confusion matrix * cross-vallidation * precission * recall * F1 score *
classification report * ROC curve * area under the curve(AUC)

\hypertarget{hyper-parameter-tuning}{%
\subsection{Hyper parameter tuning}\label{hyper-parameter-tuning}}

    \begin{tcolorbox}[breakable, size=fbox, boxrule=1pt, pad at break*=1mm,colback=cellbackground, colframe=cellborder]
\prompt{In}{incolor}{40}{\boxspacing}
\begin{Verbatim}[commandchars=\\\{\}]
\PY{c+c1}{\PYZsh{} lets\PYZsq{}s tune KNN}

\PY{n}{train\PYZus{}scores} \PY{o}{=}\PY{p}{[}\PY{p}{]}
\PY{n}{test\PYZus{}scores} \PY{o}{=} \PY{p}{[}\PY{p}{]}

\PY{c+c1}{\PYZsh{} Create a list of different values for n\PYZus{}neigbhours}
\PY{n}{neighbors} \PY{o}{=}  \PY{n+nb}{range}\PY{p}{(}\PY{l+m+mi}{1}\PY{p}{,} \PY{l+m+mi}{21}\PY{p}{)}

\PY{c+c1}{\PYZsh{} Set up KNN instance}
\PY{n}{knn} \PY{o}{=} \PY{n}{KNeighborsClassifier}\PY{p}{(}\PY{p}{)}

\PY{c+c1}{\PYZsh{}Loop through different n\PYZus{}neighboours}
\PY{k}{for} \PY{n}{i} \PY{o+ow}{in} \PY{n}{neighbors}\PY{p}{:}
    \PY{n}{knn}\PY{o}{.}\PY{n}{set\PYZus{}params}\PY{p}{(}\PY{n}{n\PYZus{}neighbors}\PY{o}{=}\PY{n}{i}\PY{p}{)}
    
    \PY{c+c1}{\PYZsh{}fit the algorithm}
    \PY{n}{knn}\PY{o}{.}\PY{n}{fit}\PY{p}{(}\PY{n}{X\PYZus{}train}\PY{p}{,} \PY{n}{y\PYZus{}train}\PY{p}{)}
    
    \PY{c+c1}{\PYZsh{}Update the training score list}
    \PY{n}{train\PYZus{}scores}\PY{o}{.}\PY{n}{append}\PY{p}{(}\PY{n}{knn}\PY{o}{.}\PY{n}{score}\PY{p}{(}\PY{n}{X\PYZus{}train}\PY{p}{,} \PY{n}{y\PYZus{}train}\PY{p}{)}\PY{p}{)}
    
    \PY{c+c1}{\PYZsh{}Update the test score list}
    \PY{n}{test\PYZus{}scores}\PY{o}{.}\PY{n}{append}\PY{p}{(}\PY{n}{knn}\PY{o}{.}\PY{n}{score}\PY{p}{(}\PY{n}{X\PYZus{}test}\PY{p}{,} \PY{n}{y\PYZus{}test}\PY{p}{)}\PY{p}{)}
    
\end{Verbatim}
\end{tcolorbox}

    \begin{tcolorbox}[breakable, size=fbox, boxrule=1pt, pad at break*=1mm,colback=cellbackground, colframe=cellborder]
\prompt{In}{incolor}{41}{\boxspacing}
\begin{Verbatim}[commandchars=\\\{\}]
\PY{n}{train\PYZus{}scores}
\end{Verbatim}
\end{tcolorbox}

            \begin{tcolorbox}[breakable, size=fbox, boxrule=.5pt, pad at break*=1mm, opacityfill=0]
\prompt{Out}{outcolor}{41}{\boxspacing}
\begin{Verbatim}[commandchars=\\\{\}]
[1.0,
 0.8099173553719008,
 0.7727272727272727,
 0.743801652892562,
 0.7603305785123967,
 0.7520661157024794,
 0.743801652892562,
 0.7231404958677686,
 0.71900826446281,
 0.6942148760330579,
 0.7272727272727273,
 0.6983471074380165,
 0.6900826446280992,
 0.6942148760330579,
 0.6859504132231405,
 0.6735537190082644,
 0.6859504132231405,
 0.6652892561983471,
 0.6818181818181818,
 0.6694214876033058]
\end{Verbatim}
\end{tcolorbox}
        
    \begin{tcolorbox}[breakable, size=fbox, boxrule=1pt, pad at break*=1mm,colback=cellbackground, colframe=cellborder]
\prompt{In}{incolor}{42}{\boxspacing}
\begin{Verbatim}[commandchars=\\\{\}]
\PY{n}{test\PYZus{}scores}
\end{Verbatim}
\end{tcolorbox}

            \begin{tcolorbox}[breakable, size=fbox, boxrule=.5pt, pad at break*=1mm, opacityfill=0]
\prompt{Out}{outcolor}{42}{\boxspacing}
\begin{Verbatim}[commandchars=\\\{\}]
[0.6229508196721312,
 0.639344262295082,
 0.6557377049180327,
 0.6721311475409836,
 0.6885245901639344,
 0.7213114754098361,
 0.7049180327868853,
 0.6885245901639344,
 0.6885245901639344,
 0.7049180327868853,
 0.7540983606557377,
 0.7377049180327869,
 0.7377049180327869,
 0.7377049180327869,
 0.6885245901639344,
 0.7213114754098361,
 0.6885245901639344,
 0.6885245901639344,
 0.7049180327868853,
 0.6557377049180327]
\end{Verbatim}
\end{tcolorbox}
        
    \begin{tcolorbox}[breakable, size=fbox, boxrule=1pt, pad at break*=1mm,colback=cellbackground, colframe=cellborder]
\prompt{In}{incolor}{44}{\boxspacing}
\begin{Verbatim}[commandchars=\\\{\}]
\PY{n}{plt}\PY{o}{.}\PY{n}{plot}\PY{p}{(}\PY{n}{neighbors}\PY{p}{,} \PY{n}{train\PYZus{}scores}\PY{p}{,} \PY{n}{label}\PY{o}{=}\PY{l+s+s2}{\PYZdq{}}\PY{l+s+s2}{train score}\PY{l+s+s2}{\PYZdq{}}\PY{p}{)}
\PY{n}{plt}\PY{o}{.}\PY{n}{plot}\PY{p}{(}\PY{n}{neighbors}\PY{p}{,} \PY{n}{test\PYZus{}scores}\PY{p}{,} \PY{n}{label}\PY{o}{=}\PY{l+s+s2}{\PYZdq{}}\PY{l+s+s2}{test score}\PY{l+s+s2}{\PYZdq{}}\PY{p}{)}
\PY{n}{plt}\PY{o}{.}\PY{n}{xticks}\PY{p}{(}\PY{n}{np}\PY{o}{.}\PY{n}{arange}\PY{p}{(}\PY{l+m+mi}{1}\PY{p}{,}\PY{l+m+mi}{21}\PY{p}{,}\PY{l+m+mi}{1}\PY{p}{)}\PY{p}{)}
\PY{n}{plt}\PY{o}{.}\PY{n}{xlabel}\PY{p}{(}\PY{l+s+s2}{\PYZdq{}}\PY{l+s+s2}{number of neighbors}\PY{l+s+s2}{\PYZdq{}}\PY{p}{)}
\PY{n}{plt}\PY{o}{.}\PY{n}{ylabel}\PY{p}{(}\PY{l+s+s2}{\PYZdq{}}\PY{l+s+s2}{model score}\PY{l+s+s2}{\PYZdq{}}\PY{p}{)}
\PY{n}{plt}\PY{o}{.}\PY{n}{legend}\PY{p}{(}\PY{p}{)}

\PY{n+nb}{print}\PY{p}{(}\PY{l+s+sa}{f}\PY{l+s+s2}{\PYZdq{}}\PY{l+s+s2}{maximum KNN score on the test data: }\PY{l+s+si}{\PYZob{}}\PY{n+nb}{max}\PY{p}{(}\PY{n}{test\PYZus{}scores}\PY{p}{)}\PY{o}{*}\PY{l+m+mi}{100}\PY{l+s+si}{:}\PY{l+s+s2}{ .2f}\PY{l+s+si}{\PYZcb{}}\PY{l+s+s2}{\PYZpc{}}\PY{l+s+s2}{\PYZdq{}}\PY{p}{)}
\end{Verbatim}
\end{tcolorbox}

    \begin{Verbatim}[commandchars=\\\{\}]
maximum KNN score on the test data:  75.41\%
    \end{Verbatim}

    \begin{center}
    \adjustimage{max size={0.9\linewidth}{0.9\paperheight}}{output_51_1.png}
    \end{center}
    { \hspace*{\fill} \\}
    
    \hypertarget{hyperparameters-tuning-with-randomizedsearchcv}{%
\subsection{Hyperparameters tuning with
RandomizedSearchCV}\label{hyperparameters-tuning-with-randomizedsearchcv}}

we are going to tune: * Logistic Regession() * RandomForestClassifier
\ldots{} using RandomizedSearchCV

    \begin{tcolorbox}[breakable, size=fbox, boxrule=1pt, pad at break*=1mm,colback=cellbackground, colframe=cellborder]
\prompt{In}{incolor}{56}{\boxspacing}
\begin{Verbatim}[commandchars=\\\{\}]
\PY{c+c1}{\PYZsh{} Cratete a hyperparameter grid for LogisticRegression }
\PY{n}{log\PYZus{}reg\PYZus{}grid} \PY{o}{=} \PY{p}{\PYZob{}}\PY{l+s+s1}{\PYZsq{}}\PY{l+s+s1}{C}\PY{l+s+s1}{\PYZsq{}}\PY{p}{:}\PY{n}{np}\PY{o}{.}\PY{n}{logspace}\PY{p}{(}\PY{o}{\PYZhy{}}\PY{l+m+mi}{4}\PY{p}{,} \PY{l+m+mi}{4}\PY{p}{,} \PY{l+m+mi}{20}\PY{p}{)}\PY{p}{,} 
               \PY{l+s+s1}{\PYZsq{}}\PY{l+s+s1}{solver}\PY{l+s+s1}{\PYZsq{}}\PY{p}{:} \PY{p}{[}\PY{l+s+s1}{\PYZsq{}}\PY{l+s+s1}{liblinear}\PY{l+s+s1}{\PYZsq{}}\PY{p}{]}\PY{p}{\PYZcb{}}

\PY{c+c1}{\PYZsh{}Create a hyperparameter grid for RandomForestClassifier }

\PY{n}{rf\PYZus{}grid} \PY{o}{=} \PY{p}{\PYZob{}}\PY{l+s+s1}{\PYZsq{}}\PY{l+s+s1}{n\PYZus{}estimators}\PY{l+s+s1}{\PYZsq{}}\PY{p}{:}\PY{n}{np}\PY{o}{.}\PY{n}{arange}\PY{p}{(}\PY{l+m+mi}{10}\PY{p}{,} \PY{l+m+mi}{1000}\PY{p}{,} \PY{l+m+mi}{50}\PY{p}{)}\PY{p}{,} 
          \PY{l+s+s1}{\PYZsq{}}\PY{l+s+s1}{max\PYZus{}depth}\PY{l+s+s1}{\PYZsq{}}\PY{p}{:} \PY{p}{[}\PY{k+kc}{None}\PY{p}{,} \PY{l+m+mi}{3}\PY{p}{,} \PY{l+m+mi}{5}\PY{p}{,} \PY{l+m+mi}{10}\PY{p}{]}\PY{p}{,} 
          \PY{l+s+s1}{\PYZsq{}}\PY{l+s+s1}{min\PYZus{}samples\PYZus{}split}\PY{l+s+s1}{\PYZsq{}}\PY{p}{:} \PY{n}{np}\PY{o}{.}\PY{n}{arange}\PY{p}{(}\PY{l+m+mi}{2}\PY{p}{,} \PY{l+m+mi}{20}\PY{p}{,} \PY{l+m+mi}{2}\PY{p}{)}\PY{p}{,}
          \PY{l+s+s1}{\PYZsq{}}\PY{l+s+s1}{min\PYZus{}samples\PYZus{}leaf}\PY{l+s+s1}{\PYZsq{}}\PY{p}{:} \PY{n}{np}\PY{o}{.}\PY{n}{arange}\PY{p}{(}\PY{l+m+mi}{1}\PY{p}{,} \PY{l+m+mi}{20}\PY{p}{,} \PY{l+m+mi}{2}\PY{p}{)}\PY{p}{\PYZcb{}}
\end{Verbatim}
\end{tcolorbox}

    Now we've got hyperparameter grids ste up for each of our models let's
tune them with randomized searchCV

    \begin{tcolorbox}[breakable, size=fbox, boxrule=1pt, pad at break*=1mm,colback=cellbackground, colframe=cellborder]
\prompt{In}{incolor}{57}{\boxspacing}
\begin{Verbatim}[commandchars=\\\{\}]
\PY{c+c1}{\PYZsh{} Tune Logistic Regression}
\PY{n}{np}\PY{o}{.}\PY{n}{random}\PY{o}{.}\PY{n}{seed}\PY{p}{(}\PY{l+m+mi}{42}\PY{p}{)}

\PY{c+c1}{\PYZsh{} Setup random hyperparameter search for Logistic Regression}
\PY{n}{rs\PYZus{}log\PYZus{}reg} \PY{o}{=} \PY{n}{RandomizedSearchCV}\PY{p}{(}\PY{n}{LogisticRegression}\PY{p}{(}\PY{p}{)}\PY{p}{,}
                               \PY{n}{param\PYZus{}distributions} \PY{o}{=} \PY{n}{log\PYZus{}reg\PYZus{}grid}\PY{p}{,}
                               \PY{n}{cv}\PY{o}{=}\PY{l+m+mi}{5}\PY{p}{,}
                               \PY{n}{n\PYZus{}iter}\PY{o}{=}\PY{l+m+mi}{20}\PY{p}{,}
                               \PY{n}{verbose}\PY{o}{=}\PY{k+kc}{True}\PY{p}{)}

\PY{c+c1}{\PYZsh{} Fit radnom hyperparameter search model for Logistic Regression}
\PY{n}{rs\PYZus{}log\PYZus{}reg}\PY{o}{.}\PY{n}{fit}\PY{p}{(}\PY{n}{X\PYZus{}train}\PY{p}{,} \PY{n}{y\PYZus{}train}\PY{p}{)}
\end{Verbatim}
\end{tcolorbox}

    \begin{Verbatim}[commandchars=\\\{\}]
Fitting 5 folds for each of 20 candidates, totalling 100 fits
    \end{Verbatim}

    \begin{Verbatim}[commandchars=\\\{\}]
[Parallel(n\_jobs=1)]: Using backend SequentialBackend with 1 concurrent workers.
[Parallel(n\_jobs=1)]: Done 100 out of 100 | elapsed:    0.9s finished
    \end{Verbatim}

            \begin{tcolorbox}[breakable, size=fbox, boxrule=.5pt, pad at break*=1mm, opacityfill=0]
\prompt{Out}{outcolor}{57}{\boxspacing}
\begin{Verbatim}[commandchars=\\\{\}]
RandomizedSearchCV(cv=5, error\_score=nan,
                   estimator=LogisticRegression(C=1.0, class\_weight=None,
                                                dual=False, fit\_intercept=True,
                                                intercept\_scaling=1,
                                                l1\_ratio=None, max\_iter=100,
                                                multi\_class='auto', n\_jobs=None,
                                                penalty='l2', random\_state=None,
                                                solver='lbfgs', tol=0.0001,
                                                verbose=0, warm\_start=False),
                   iid='deprecated', n\_iter=20, n\_jobs=None,
                   param\_distributions=\{'C':{\ldots}
       4.83293024e-03, 1.27427499e-02, 3.35981829e-02, 8.85866790e-02,
       2.33572147e-01, 6.15848211e-01, 1.62377674e+00, 4.28133240e+00,
       1.12883789e+01, 2.97635144e+01, 7.84759970e+01, 2.06913808e+02,
       5.45559478e+02, 1.43844989e+03, 3.79269019e+03, 1.00000000e+04]),
                                        'solver': ['liblinear']\},
                   pre\_dispatch='2*n\_jobs', random\_state=None, refit=True,
                   return\_train\_score=False, scoring=None, verbose=True)
\end{Verbatim}
\end{tcolorbox}
        
    \begin{tcolorbox}[breakable, size=fbox, boxrule=1pt, pad at break*=1mm,colback=cellbackground, colframe=cellborder]
\prompt{In}{incolor}{58}{\boxspacing}
\begin{Verbatim}[commandchars=\\\{\}]
\PY{n}{rs\PYZus{}log\PYZus{}reg}\PY{o}{.}\PY{n}{best\PYZus{}params\PYZus{}}
\end{Verbatim}
\end{tcolorbox}

            \begin{tcolorbox}[breakable, size=fbox, boxrule=.5pt, pad at break*=1mm, opacityfill=0]
\prompt{Out}{outcolor}{58}{\boxspacing}
\begin{Verbatim}[commandchars=\\\{\}]
\{'solver': 'liblinear', 'C': 0.23357214690901212\}
\end{Verbatim}
\end{tcolorbox}
        
    \begin{tcolorbox}[breakable, size=fbox, boxrule=1pt, pad at break*=1mm,colback=cellbackground, colframe=cellborder]
\prompt{In}{incolor}{59}{\boxspacing}
\begin{Verbatim}[commandchars=\\\{\}]
\PY{n}{rs\PYZus{}log\PYZus{}reg}\PY{o}{.}\PY{n}{score}\PY{p}{(}\PY{n}{X\PYZus{}test}\PY{p}{,} \PY{n}{y\PYZus{}test}\PY{p}{)}
\end{Verbatim}
\end{tcolorbox}

            \begin{tcolorbox}[breakable, size=fbox, boxrule=.5pt, pad at break*=1mm, opacityfill=0]
\prompt{Out}{outcolor}{59}{\boxspacing}
\begin{Verbatim}[commandchars=\\\{\}]
0.8852459016393442
\end{Verbatim}
\end{tcolorbox}
        
    noow we've tuned LOgisticRegression() let's do the same for random
forest classifier

    \begin{tcolorbox}[breakable, size=fbox, boxrule=1pt, pad at break*=1mm,colback=cellbackground, colframe=cellborder]
\prompt{In}{incolor}{62}{\boxspacing}
\begin{Verbatim}[commandchars=\\\{\}]
\PY{c+c1}{\PYZsh{} setup random seed}
\PY{n}{np}\PY{o}{.}\PY{n}{random}\PY{o}{.}\PY{n}{seed}\PY{p}{(}\PY{l+m+mi}{42}\PY{p}{)}

\PY{c+c1}{\PYZsh{}Setup random parameter for RandomForestClassifier()}

\PY{n}{rs\PYZus{}rf} \PY{o}{=} \PY{n}{RandomizedSearchCV}\PY{p}{(}\PY{n}{RandomForestClassifier}\PY{p}{(}\PY{p}{)}\PY{p}{,}
                          \PY{n}{param\PYZus{}distributions} \PY{o}{=}\PY{n}{rf\PYZus{}grid}\PY{p}{,}
                          \PY{n}{cv}\PY{o}{=}\PY{l+m+mi}{5}\PY{p}{,}
                          \PY{n}{n\PYZus{}iter}\PY{o}{=}\PY{l+m+mi}{20}\PY{p}{,}
                          \PY{n}{verbose}\PY{o}{=}\PY{k+kc}{True}\PY{p}{)}

\PY{c+c1}{\PYZsh{} Fit radom hyperparameter search model for RandomForrestClassifier()}
\PY{n}{rs\PYZus{}rf}\PY{o}{.}\PY{n}{fit}\PY{p}{(}\PY{n}{X\PYZus{}train}\PY{p}{,} \PY{n}{y\PYZus{}train}\PY{p}{)}
\end{Verbatim}
\end{tcolorbox}

    \begin{Verbatim}[commandchars=\\\{\}]
Fitting 5 folds for each of 20 candidates, totalling 100 fits
    \end{Verbatim}

    \begin{Verbatim}[commandchars=\\\{\}]
[Parallel(n\_jobs=1)]: Using backend SequentialBackend with 1 concurrent workers.
[Parallel(n\_jobs=1)]: Done 100 out of 100 | elapsed:  3.0min finished
    \end{Verbatim}

            \begin{tcolorbox}[breakable, size=fbox, boxrule=.5pt, pad at break*=1mm, opacityfill=0]
\prompt{Out}{outcolor}{62}{\boxspacing}
\begin{Verbatim}[commandchars=\\\{\}]
RandomizedSearchCV(cv=5, error\_score=nan,
                   estimator=RandomForestClassifier(bootstrap=True,
                                                    ccp\_alpha=0.0,
                                                    class\_weight=None,
                                                    criterion='gini',
                                                    max\_depth=None,
                                                    max\_features='auto',
                                                    max\_leaf\_nodes=None,
                                                    max\_samples=None,
                                                    min\_impurity\_decrease=0.0,
                                                    min\_impurity\_split=None,
                                                    min\_samples\_leaf=1,
                                                    min\_samples\_split=2,
min\_weight\_fraction\_leaf=0.0,
                                                    n\_estimators=100,
                                                    n\_jobs{\ldots}
                   param\_distributions=\{'max\_depth': [None, 3, 5, 10],
                                        'min\_samples\_leaf': array([ 1,  3,  5,
7,  9, 11, 13, 15, 17, 19]),
                                        'min\_samples\_split': array([ 2,  4,  6,
8, 10, 12, 14, 16, 18]),
                                        'n\_estimators': array([ 10,  60, 110,
160, 210, 260, 310, 360, 410, 460, 510, 560, 610,
       660, 710, 760, 810, 860, 910, 960])\},
                   pre\_dispatch='2*n\_jobs', random\_state=None, refit=True,
                   return\_train\_score=False, scoring=None, verbose=True)
\end{Verbatim}
\end{tcolorbox}
        
    \begin{tcolorbox}[breakable, size=fbox, boxrule=1pt, pad at break*=1mm,colback=cellbackground, colframe=cellborder]
\prompt{In}{incolor}{63}{\boxspacing}
\begin{Verbatim}[commandchars=\\\{\}]
\PY{c+c1}{\PYZsh{} Find the hyper parameters }
\PY{n}{rs\PYZus{}rf}\PY{o}{.}\PY{n}{best\PYZus{}params\PYZus{}}
\end{Verbatim}
\end{tcolorbox}

            \begin{tcolorbox}[breakable, size=fbox, boxrule=.5pt, pad at break*=1mm, opacityfill=0]
\prompt{Out}{outcolor}{63}{\boxspacing}
\begin{Verbatim}[commandchars=\\\{\}]
\{'n\_estimators': 210,
 'min\_samples\_split': 4,
 'min\_samples\_leaf': 19,
 'max\_depth': 3\}
\end{Verbatim}
\end{tcolorbox}
        
    \begin{tcolorbox}[breakable, size=fbox, boxrule=1pt, pad at break*=1mm,colback=cellbackground, colframe=cellborder]
\prompt{In}{incolor}{64}{\boxspacing}
\begin{Verbatim}[commandchars=\\\{\}]
\PY{c+c1}{\PYZsh{} Evaluate the randomized search Random Forest Classifier model }
\PY{n}{rs\PYZus{}rf}\PY{o}{.}\PY{n}{score}\PY{p}{(}\PY{n}{X\PYZus{}test}\PY{p}{,} \PY{n}{y\PYZus{}test}\PY{p}{)}
\end{Verbatim}
\end{tcolorbox}

            \begin{tcolorbox}[breakable, size=fbox, boxrule=.5pt, pad at break*=1mm, opacityfill=0]
\prompt{Out}{outcolor}{64}{\boxspacing}
\begin{Verbatim}[commandchars=\\\{\}]
0.8688524590163934
\end{Verbatim}
\end{tcolorbox}
        
    \begin{tcolorbox}[breakable, size=fbox, boxrule=1pt, pad at break*=1mm,colback=cellbackground, colframe=cellborder]
\prompt{In}{incolor}{67}{\boxspacing}
\begin{Verbatim}[commandchars=\\\{\}]
\PY{n}{models\PYZus{}score}
\end{Verbatim}
\end{tcolorbox}

            \begin{tcolorbox}[breakable, size=fbox, boxrule=.5pt, pad at break*=1mm, opacityfill=0]
\prompt{Out}{outcolor}{67}{\boxspacing}
\begin{Verbatim}[commandchars=\\\{\}]
\{'Logistic Regression': 0.8852459016393442,
 'KNN': 0.6885245901639344,
 'Random Forest': 0.8360655737704918\}
\end{Verbatim}
\end{tcolorbox}
        
    Ways to tun hyperparameters

\begin{enumerate}
\def\labelenumi{\arabic{enumi}.}
\tightlist
\item
  by hand
\item
  RandomizedSearchCV
\item
  GridSearch
\end{enumerate}

    \hypertarget{hyperparameter-tuning-with-gridsearchcv}{%
\subsection{Hyperparameter Tuning with
GridSearchCV}\label{hyperparameter-tuning-with-gridsearchcv}}

since our LogisticRegresion model provides the best scores we will use
GridSearchCV to imptove it

    \begin{tcolorbox}[breakable, size=fbox, boxrule=1pt, pad at break*=1mm,colback=cellbackground, colframe=cellborder]
\prompt{In}{incolor}{68}{\boxspacing}
\begin{Verbatim}[commandchars=\\\{\}]
\PY{c+c1}{\PYZsh{}Deifferent hyperparamters for our logisticRegerssion model }
\PY{n}{log\PYZus{}reg\PYZus{}grid} \PY{o}{=} \PY{p}{\PYZob{}}\PY{l+s+s1}{\PYZsq{}}\PY{l+s+s1}{C}\PY{l+s+s1}{\PYZsq{}}\PY{p}{:} \PY{n}{np}\PY{o}{.}\PY{n}{logspace}\PY{p}{(}\PY{o}{\PYZhy{}}\PY{l+m+mi}{4}\PY{p}{,} \PY{l+m+mi}{4}\PY{p}{,} \PY{l+m+mi}{30}\PY{p}{)}\PY{p}{,}
               \PY{l+s+s1}{\PYZsq{}}\PY{l+s+s1}{solver}\PY{l+s+s1}{\PYZsq{}}\PY{p}{:} \PY{p}{[}\PY{l+s+s1}{\PYZsq{}}\PY{l+s+s1}{liblinear}\PY{l+s+s1}{\PYZsq{}}\PY{p}{]}\PY{p}{\PYZcb{}}
\PY{c+c1}{\PYZsh{}setup grid hyperparameter seach for logistic regression}
\PY{n}{gs\PYZus{}log\PYZus{}reg} \PY{o}{=} \PY{n}{GridSearchCV}\PY{p}{(}\PY{n}{LogisticRegression}\PY{p}{(}\PY{p}{)}\PY{p}{,}
                         \PY{n}{param\PYZus{}grid} \PY{o}{=} \PY{n}{log\PYZus{}reg\PYZus{}grid}\PY{p}{,} 
                         \PY{n}{cv}\PY{o}{=} \PY{l+m+mi}{5}\PY{p}{,} 
                         \PY{n}{verbose} \PY{o}{=} \PY{k+kc}{True}\PY{p}{)}

\PY{c+c1}{\PYZsh{}fit grid hyperparameter search model }

\PY{n}{gs\PYZus{}log\PYZus{}reg}\PY{o}{.}\PY{n}{fit}\PY{p}{(}\PY{n}{X\PYZus{}train}\PY{p}{,} \PY{n}{y\PYZus{}train}\PY{p}{)}\PY{p}{;}
\end{Verbatim}
\end{tcolorbox}

    \begin{Verbatim}[commandchars=\\\{\}]
Fitting 5 folds for each of 30 candidates, totalling 150 fits
    \end{Verbatim}

    \begin{Verbatim}[commandchars=\\\{\}]
[Parallel(n\_jobs=1)]: Using backend SequentialBackend with 1 concurrent workers.
[Parallel(n\_jobs=1)]: Done 150 out of 150 | elapsed:    5.3s finished
    \end{Verbatim}

    \begin{tcolorbox}[breakable, size=fbox, boxrule=1pt, pad at break*=1mm,colback=cellbackground, colframe=cellborder]
\prompt{In}{incolor}{69}{\boxspacing}
\begin{Verbatim}[commandchars=\\\{\}]
\PY{c+c1}{\PYZsh{} chek the best hyperparamete}
\PY{n}{gs\PYZus{}log\PYZus{}reg}\PY{o}{.}\PY{n}{best\PYZus{}params\PYZus{}}
\end{Verbatim}
\end{tcolorbox}

            \begin{tcolorbox}[breakable, size=fbox, boxrule=.5pt, pad at break*=1mm, opacityfill=0]
\prompt{Out}{outcolor}{69}{\boxspacing}
\begin{Verbatim}[commandchars=\\\{\}]
\{'C': 0.20433597178569418, 'solver': 'liblinear'\}
\end{Verbatim}
\end{tcolorbox}
        
    \begin{tcolorbox}[breakable, size=fbox, boxrule=1pt, pad at break*=1mm,colback=cellbackground, colframe=cellborder]
\prompt{In}{incolor}{70}{\boxspacing}
\begin{Verbatim}[commandchars=\\\{\}]
\PY{n}{gs\PYZus{}log\PYZus{}reg}\PY{o}{.}\PY{n}{score}\PY{p}{(}\PY{n}{X\PYZus{}test}\PY{p}{,} \PY{n}{y\PYZus{}test}\PY{p}{)}
\end{Verbatim}
\end{tcolorbox}

            \begin{tcolorbox}[breakable, size=fbox, boxrule=.5pt, pad at break*=1mm, opacityfill=0]
\prompt{Out}{outcolor}{70}{\boxspacing}
\begin{Verbatim}[commandchars=\\\{\}]
0.8852459016393442
\end{Verbatim}
\end{tcolorbox}
        
    \hypertarget{evauating-our-tuned-machine-learning-classifier-beyond-accuracy}{%
\subsection{Evauating our tuned machine learning classifier beyond
accuracy}\label{evauating-our-tuned-machine-learning-classifier-beyond-accuracy}}

\begin{itemize}
\tightlist
\item
  ROC curve and AUC score
\item
  Confusion matrix
\item
  Precision
\item
  Recall
\item
  F1-score (cros validation if possible)
\end{itemize}

t make comparison and evaluate trained model, first we need to make
predictions

    \begin{tcolorbox}[breakable, size=fbox, boxrule=1pt, pad at break*=1mm,colback=cellbackground, colframe=cellborder]
\prompt{In}{incolor}{72}{\boxspacing}
\begin{Verbatim}[commandchars=\\\{\}]
\PY{c+c1}{\PYZsh{} Make predictions with tuned model }
\PY{n}{y\PYZus{}preds} \PY{o}{=} \PY{n}{gs\PYZus{}log\PYZus{}reg}\PY{o}{.}\PY{n}{predict}\PY{p}{(}\PY{n}{X\PYZus{}test}\PY{p}{)}
\end{Verbatim}
\end{tcolorbox}

    \begin{tcolorbox}[breakable, size=fbox, boxrule=1pt, pad at break*=1mm,colback=cellbackground, colframe=cellborder]
\prompt{In}{incolor}{73}{\boxspacing}
\begin{Verbatim}[commandchars=\\\{\}]
\PY{n}{y\PYZus{}test}
\end{Verbatim}
\end{tcolorbox}

            \begin{tcolorbox}[breakable, size=fbox, boxrule=.5pt, pad at break*=1mm, opacityfill=0]
\prompt{Out}{outcolor}{73}{\boxspacing}
\begin{Verbatim}[commandchars=\\\{\}]
179    0
228    0
111    1
246    0
60     1
      ..
249    0
104    1
300    0
193    0
184    0
Name: target, Length: 61, dtype: int64
\end{Verbatim}
\end{tcolorbox}
        
    \begin{tcolorbox}[breakable, size=fbox, boxrule=1pt, pad at break*=1mm,colback=cellbackground, colframe=cellborder]
\prompt{In}{incolor}{76}{\boxspacing}
\begin{Verbatim}[commandchars=\\\{\}]
\PY{n}{y\PYZus{}preds}
\end{Verbatim}
\end{tcolorbox}

            \begin{tcolorbox}[breakable, size=fbox, boxrule=.5pt, pad at break*=1mm, opacityfill=0]
\prompt{Out}{outcolor}{76}{\boxspacing}
\begin{Verbatim}[commandchars=\\\{\}]
array([0, 1, 1, 0, 1, 1, 1, 0, 0, 1, 1, 0, 1, 0, 1, 1, 1, 0, 0, 0, 1, 0,
       0, 1, 1, 1, 1, 1, 0, 1, 0, 0, 0, 0, 1, 0, 1, 1, 1, 1, 1, 1, 1, 1,
       1, 0, 1, 1, 0, 0, 0, 0, 1, 1, 0, 0, 0, 1, 0, 0, 0], dtype=int64)
\end{Verbatim}
\end{tcolorbox}
        
    \begin{tcolorbox}[breakable, size=fbox, boxrule=1pt, pad at break*=1mm,colback=cellbackground, colframe=cellborder]
\prompt{In}{incolor}{77}{\boxspacing}
\begin{Verbatim}[commandchars=\\\{\}]
\PY{c+c1}{\PYZsh{} Plot ROC curve and calculate AUC metric}
\PY{n}{plot\PYZus{}roc\PYZus{}curve}\PY{p}{(}\PY{n}{gs\PYZus{}log\PYZus{}reg}\PY{p}{,} \PY{n}{X\PYZus{}test}\PY{p}{,} \PY{n}{y\PYZus{}test}\PY{p}{)}
\end{Verbatim}
\end{tcolorbox}

            \begin{tcolorbox}[breakable, size=fbox, boxrule=.5pt, pad at break*=1mm, opacityfill=0]
\prompt{Out}{outcolor}{77}{\boxspacing}
\begin{Verbatim}[commandchars=\\\{\}]
<sklearn.metrics.\_plot.roc\_curve.RocCurveDisplay at 0x1c1849e2580>
\end{Verbatim}
\end{tcolorbox}
        
    \begin{center}
    \adjustimage{max size={0.9\linewidth}{0.9\paperheight}}{output_72_1.png}
    \end{center}
    { \hspace*{\fill} \\}
    
    \begin{tcolorbox}[breakable, size=fbox, boxrule=1pt, pad at break*=1mm,colback=cellbackground, colframe=cellborder]
\prompt{In}{incolor}{78}{\boxspacing}
\begin{Verbatim}[commandchars=\\\{\}]
\PY{c+c1}{\PYZsh{} Confusion matrix }
\PY{n+nb}{print}\PY{p}{(}\PY{n}{confusion\PYZus{}matrix}\PY{p}{(}\PY{n}{y\PYZus{}test}\PY{p}{,} \PY{n}{y\PYZus{}preds}\PY{p}{)}\PY{p}{)}
\end{Verbatim}
\end{tcolorbox}

    \begin{Verbatim}[commandchars=\\\{\}]
[[25  4]
 [ 3 29]]
    \end{Verbatim}

    \begin{tcolorbox}[breakable, size=fbox, boxrule=1pt, pad at break*=1mm,colback=cellbackground, colframe=cellborder]
\prompt{In}{incolor}{82}{\boxspacing}
\begin{Verbatim}[commandchars=\\\{\}]
\PY{n}{sns}\PY{o}{.}\PY{n}{set}\PY{p}{(}\PY{n}{font\PYZus{}scale}\PY{o}{=}\PY{l+m+mf}{1.5}\PY{p}{)}

\PY{k}{def} \PY{n+nf}{plot\PYZus{}conf\PYZus{}mat}\PY{p}{(}\PY{n}{y\PYZus{}test}\PY{p}{,} \PY{n}{y\PYZus{}preds}\PY{p}{)}\PY{p}{:}
    \PY{l+s+sd}{\PYZdq{}\PYZdq{}\PYZdq{} Plots an nice looking confucion matrix using Seaborn\PYZsq{}s heatmap()}
\PY{l+s+sd}{    \PYZdq{}\PYZdq{}\PYZdq{}}
    \PY{n}{fig}\PY{p}{,} \PY{n}{ax} \PY{o}{=}\PY{n}{plt}\PY{o}{.}\PY{n}{subplots}\PY{p}{(}\PY{n}{figsize}\PY{o}{=}\PY{p}{(}\PY{l+m+mi}{3}\PY{p}{,}\PY{l+m+mi}{3}\PY{p}{)}\PY{p}{)}
    \PY{n}{ax} \PY{o}{=} \PY{n}{sns}\PY{o}{.}\PY{n}{heatmap}\PY{p}{(}\PY{n}{confusion\PYZus{}matrix}\PY{p}{(}\PY{n}{y\PYZus{}test}\PY{p}{,} \PY{n}{y\PYZus{}preds}\PY{p}{)}\PY{p}{,}
                    \PY{n}{annot}\PY{o}{=}\PY{k+kc}{True}\PY{p}{,}
                    \PY{n}{cbar}\PY{o}{=}\PY{k+kc}{False}\PY{p}{)}
    \PY{n}{plt}\PY{o}{.}\PY{n}{xlabel}\PY{p}{(}\PY{l+s+s2}{\PYZdq{}}\PY{l+s+s2}{true label}\PY{l+s+s2}{\PYZdq{}}\PY{p}{)}
    \PY{n}{plt}\PY{o}{.}\PY{n}{ylabel}\PY{p}{(}\PY{l+s+s2}{\PYZdq{}}\PY{l+s+s2}{predicted label}\PY{l+s+s2}{\PYZdq{}}\PY{p}{)}
    
\PY{n}{plot\PYZus{}conf\PYZus{}mat}\PY{p}{(} \PY{n}{y\PYZus{}test}\PY{p}{,} \PY{n}{y\PYZus{}preds}\PY{p}{)}
\end{Verbatim}
\end{tcolorbox}

    \begin{center}
    \adjustimage{max size={0.9\linewidth}{0.9\paperheight}}{output_74_0.png}
    \end{center}
    { \hspace*{\fill} \\}
    
    now we've got a ROC and AUC and aconfusino atrix, lets geta a
classification repport as well as a cross-validated precision, recall
and f1 score

    \begin{tcolorbox}[breakable, size=fbox, boxrule=1pt, pad at break*=1mm,colback=cellbackground, colframe=cellborder]
\prompt{In}{incolor}{83}{\boxspacing}
\begin{Verbatim}[commandchars=\\\{\}]
\PY{n+nb}{print}\PY{p}{(}\PY{n}{classification\PYZus{}report}\PY{p}{(}\PY{n}{y\PYZus{}test}\PY{p}{,} \PY{n}{y\PYZus{}preds}\PY{p}{)}\PY{p}{)}
\end{Verbatim}
\end{tcolorbox}

    \begin{Verbatim}[commandchars=\\\{\}]
              precision    recall  f1-score   support

           0       0.89      0.86      0.88        29
           1       0.88      0.91      0.89        32

    accuracy                           0.89        61
   macro avg       0.89      0.88      0.88        61
weighted avg       0.89      0.89      0.89        61

    \end{Verbatim}

    \hypertarget{calculate-evaluation-matrix-using-cross-validation}{%
\subsubsection{Calculate evaluation matrix using
cross-validation}\label{calculate-evaluation-matrix-using-cross-validation}}

we're going to calculate precision, recall and f1 scores using
cross-validation

    \begin{tcolorbox}[breakable, size=fbox, boxrule=1pt, pad at break*=1mm,colback=cellbackground, colframe=cellborder]
\prompt{In}{incolor}{84}{\boxspacing}
\begin{Verbatim}[commandchars=\\\{\}]
\PY{c+c1}{\PYZsh{}check best hyperparameters }
\PY{n}{gs\PYZus{}log\PYZus{}reg}\PY{o}{.}\PY{n}{best\PYZus{}params\PYZus{}}
\end{Verbatim}
\end{tcolorbox}

            \begin{tcolorbox}[breakable, size=fbox, boxrule=.5pt, pad at break*=1mm, opacityfill=0]
\prompt{Out}{outcolor}{84}{\boxspacing}
\begin{Verbatim}[commandchars=\\\{\}]
\{'C': 0.20433597178569418, 'solver': 'liblinear'\}
\end{Verbatim}
\end{tcolorbox}
        
    \begin{tcolorbox}[breakable, size=fbox, boxrule=1pt, pad at break*=1mm,colback=cellbackground, colframe=cellborder]
\prompt{In}{incolor}{85}{\boxspacing}
\begin{Verbatim}[commandchars=\\\{\}]
\PY{c+c1}{\PYZsh{}create a new classifier with best paraeteres }
\PY{n}{clf} \PY{o}{=} \PY{n}{LogisticRegression} \PY{p}{(}\PY{n}{C} \PY{o}{=}  \PY{l+m+mf}{0.20433597178569418}\PY{p}{,} 
                         \PY{n}{solver} \PY{o}{=} \PY{l+s+s1}{\PYZsq{}}\PY{l+s+s1}{liblinear}\PY{l+s+s1}{\PYZsq{}}\PY{p}{)}
\end{Verbatim}
\end{tcolorbox}

    \begin{tcolorbox}[breakable, size=fbox, boxrule=1pt, pad at break*=1mm,colback=cellbackground, colframe=cellborder]
\prompt{In}{incolor}{86}{\boxspacing}
\begin{Verbatim}[commandchars=\\\{\}]
\PY{c+c1}{\PYZsh{}cross\PYZhy{}validates accuracy }
\PY{n}{cv\PYZus{}acc} \PY{o}{=} \PY{n}{cross\PYZus{}val\PYZus{}score}\PY{p}{(}\PY{n}{clf}\PY{p}{,} 
                        \PY{n}{X}\PY{p}{,} 
                        \PY{n}{y}\PY{p}{,} 
                        \PY{n}{cv} \PY{o}{=} \PY{l+m+mi}{5}\PY{p}{,} 
                        \PY{n}{scoring}\PY{o}{=} \PY{l+s+s1}{\PYZsq{}}\PY{l+s+s1}{accuracy}\PY{l+s+s1}{\PYZsq{}}\PY{p}{)}

\PY{n}{cv\PYZus{}acc}
\end{Verbatim}
\end{tcolorbox}

            \begin{tcolorbox}[breakable, size=fbox, boxrule=.5pt, pad at break*=1mm, opacityfill=0]
\prompt{Out}{outcolor}{86}{\boxspacing}
\begin{Verbatim}[commandchars=\\\{\}]
array([0.81967213, 0.90163934, 0.86885246, 0.88333333, 0.75      ])
\end{Verbatim}
\end{tcolorbox}
        
    \begin{tcolorbox}[breakable, size=fbox, boxrule=1pt, pad at break*=1mm,colback=cellbackground, colframe=cellborder]
\prompt{In}{incolor}{89}{\boxspacing}
\begin{Verbatim}[commandchars=\\\{\}]
\PY{n}{cv\PYZus{}acc} \PY{o}{=} \PY{n}{np}\PY{o}{.}\PY{n}{mean}\PY{p}{(}\PY{n}{cv\PYZus{}acc}\PY{p}{)}
\PY{n}{cv\PYZus{}acc}
\end{Verbatim}
\end{tcolorbox}

            \begin{tcolorbox}[breakable, size=fbox, boxrule=.5pt, pad at break*=1mm, opacityfill=0]
\prompt{Out}{outcolor}{89}{\boxspacing}
\begin{Verbatim}[commandchars=\\\{\}]
0.8446994535519124
\end{Verbatim}
\end{tcolorbox}
        
    \begin{tcolorbox}[breakable, size=fbox, boxrule=1pt, pad at break*=1mm,colback=cellbackground, colframe=cellborder]
\prompt{In}{incolor}{90}{\boxspacing}
\begin{Verbatim}[commandchars=\\\{\}]
\PY{n}{cv\PYZus{}precision} \PY{o}{=} \PY{n}{cross\PYZus{}val\PYZus{}score}\PY{p}{(}\PY{n}{clf}\PY{p}{,} 
                        \PY{n}{X}\PY{p}{,} 
                        \PY{n}{y}\PY{p}{,} 
                        \PY{n}{cv} \PY{o}{=} \PY{l+m+mi}{5}\PY{p}{,} 
                        \PY{n}{scoring}\PY{o}{=} \PY{l+s+s1}{\PYZsq{}}\PY{l+s+s1}{precision}\PY{l+s+s1}{\PYZsq{}}\PY{p}{)}

\PY{n}{cv\PYZus{}precision} \PY{o}{=} \PY{n}{np}\PY{o}{.}\PY{n}{mean}\PY{p}{(}\PY{n}{cv\PYZus{}precision}\PY{p}{)}
\PY{n}{cv\PYZus{}precision}
\end{Verbatim}
\end{tcolorbox}

            \begin{tcolorbox}[breakable, size=fbox, boxrule=.5pt, pad at break*=1mm, opacityfill=0]
\prompt{Out}{outcolor}{90}{\boxspacing}
\begin{Verbatim}[commandchars=\\\{\}]
0.8207936507936507
\end{Verbatim}
\end{tcolorbox}
        
    \begin{tcolorbox}[breakable, size=fbox, boxrule=1pt, pad at break*=1mm,colback=cellbackground, colframe=cellborder]
\prompt{In}{incolor}{93}{\boxspacing}
\begin{Verbatim}[commandchars=\\\{\}]
\PY{n}{cv\PYZus{}recall} \PY{o}{=} \PY{n}{cross\PYZus{}val\PYZus{}score}\PY{p}{(}\PY{n}{clf}\PY{p}{,} 
                        \PY{n}{X}\PY{p}{,} 
                        \PY{n}{y}\PY{p}{,} 
                        \PY{n}{cv} \PY{o}{=} \PY{l+m+mi}{5}\PY{p}{,} 
                        \PY{n}{scoring}\PY{o}{=} \PY{l+s+s1}{\PYZsq{}}\PY{l+s+s1}{recall}\PY{l+s+s1}{\PYZsq{}}\PY{p}{)}

\PY{n}{cv\PYZus{}recall} \PY{o}{=} \PY{n}{np}\PY{o}{.}\PY{n}{mean}\PY{p}{(}\PY{n}{cv\PYZus{}recall}\PY{p}{)}
\PY{n}{cv\PYZus{}recall}
\end{Verbatim}
\end{tcolorbox}

            \begin{tcolorbox}[breakable, size=fbox, boxrule=.5pt, pad at break*=1mm, opacityfill=0]
\prompt{Out}{outcolor}{93}{\boxspacing}
\begin{Verbatim}[commandchars=\\\{\}]
0.9212121212121213
\end{Verbatim}
\end{tcolorbox}
        
    \begin{tcolorbox}[breakable, size=fbox, boxrule=1pt, pad at break*=1mm,colback=cellbackground, colframe=cellborder]
\prompt{In}{incolor}{94}{\boxspacing}
\begin{Verbatim}[commandchars=\\\{\}]
\PY{n}{cv\PYZus{}f1} \PY{o}{=} \PY{n}{cross\PYZus{}val\PYZus{}score}\PY{p}{(}\PY{n}{clf}\PY{p}{,} 
                        \PY{n}{X}\PY{p}{,} 
                        \PY{n}{y}\PY{p}{,} 
                        \PY{n}{cv} \PY{o}{=} \PY{l+m+mi}{5}\PY{p}{,} 
                        \PY{n}{scoring}\PY{o}{=} \PY{l+s+s1}{\PYZsq{}}\PY{l+s+s1}{f1}\PY{l+s+s1}{\PYZsq{}}\PY{p}{)}

\PY{n}{cv\PYZus{}f1} \PY{o}{=} \PY{n}{np}\PY{o}{.}\PY{n}{mean}\PY{p}{(}\PY{n}{cv\PYZus{}f1}\PY{p}{)}
\PY{n}{cv\PYZus{}f1}
\end{Verbatim}
\end{tcolorbox}

            \begin{tcolorbox}[breakable, size=fbox, boxrule=.5pt, pad at break*=1mm, opacityfill=0]
\prompt{Out}{outcolor}{94}{\boxspacing}
\begin{Verbatim}[commandchars=\\\{\}]
0.8673007976269721
\end{Verbatim}
\end{tcolorbox}
        
    \begin{tcolorbox}[breakable, size=fbox, boxrule=1pt, pad at break*=1mm,colback=cellbackground, colframe=cellborder]
\prompt{In}{incolor}{96}{\boxspacing}
\begin{Verbatim}[commandchars=\\\{\}]
\PY{c+c1}{\PYZsh{} create a plot }
\PY{n}{cv\PYZus{}metrics} \PY{o}{=} \PY{n}{pd}\PY{o}{.}\PY{n}{DataFrame}\PY{p}{(}\PY{p}{\PYZob{}}\PY{l+s+s1}{\PYZsq{}}\PY{l+s+s1}{Accuracy}\PY{l+s+s1}{\PYZsq{}}\PY{p}{:} \PY{n}{cv\PYZus{}acc}\PY{p}{,} 
                          \PY{l+s+s1}{\PYZsq{}}\PY{l+s+s1}{Precision}\PY{l+s+s1}{\PYZsq{}}\PY{p}{:} \PY{n}{cv\PYZus{}precision}\PY{p}{,} 
                          \PY{l+s+s1}{\PYZsq{}}\PY{l+s+s1}{Recall}\PY{l+s+s1}{\PYZsq{}}\PY{p}{:} \PY{n}{cv\PYZus{}recall}\PY{p}{,}
                          \PY{l+s+s1}{\PYZsq{}}\PY{l+s+s1}{F1}\PY{l+s+s1}{\PYZsq{}}\PY{p}{:}\PY{n}{cv\PYZus{}f1}\PY{p}{\PYZcb{}}\PY{p}{,} 
                         \PY{n}{index}\PY{o}{=}\PY{p}{[}\PY{l+m+mi}{0}\PY{p}{]}\PY{p}{)}
\end{Verbatim}
\end{tcolorbox}

    \begin{tcolorbox}[breakable, size=fbox, boxrule=1pt, pad at break*=1mm,colback=cellbackground, colframe=cellborder]
\prompt{In}{incolor}{98}{\boxspacing}
\begin{Verbatim}[commandchars=\\\{\}]
\PY{n}{cv\PYZus{}metrics}\PY{o}{.}\PY{n}{T}\PY{o}{.}\PY{n}{plot}\PY{o}{.}\PY{n}{bar}\PY{p}{(}\PY{n}{title} \PY{o}{=} \PY{l+s+s1}{\PYZsq{}}\PY{l+s+s1}{Cross\PYZhy{}validated classifictacyionmetrics}\PY{l+s+s1}{\PYZsq{}}\PY{p}{,} \PY{n}{legend} \PY{o}{=} \PY{k+kc}{False}\PY{p}{)}\PY{p}{;}
\end{Verbatim}
\end{tcolorbox}

    \begin{center}
    \adjustimage{max size={0.9\linewidth}{0.9\paperheight}}{output_86_0.png}
    \end{center}
    { \hspace*{\fill} \\}
    
    \hypertarget{feature-importance}{%
\subsubsection{Feature importance}\label{feature-importance}}

Feature importance is another way of asking , ``Which features
contribute most to the outcomes of the model and how did they
contribute?''

Finding feature importance is differenet for each machine learning
model. One way to find feature importace is to search for ``(model name)
feature importance''

Let's find feature importance for pur Lpgistic Regression model

    \begin{tcolorbox}[breakable, size=fbox, boxrule=1pt, pad at break*=1mm,colback=cellbackground, colframe=cellborder]
\prompt{In}{incolor}{102}{\boxspacing}
\begin{Verbatim}[commandchars=\\\{\}]
\PY{c+c1}{\PYZsh{} Fit an instance of Logistic Regression }
\PY{n}{gs\PYZus{}log\PYZus{}reg}\PY{o}{.}\PY{n}{best\PYZus{}params\PYZus{}}
\PY{n}{clf} \PY{o}{=} \PY{n}{LogisticRegression}\PY{p}{(}\PY{n}{C}\PY{o}{=}\PY{l+m+mf}{0.20433597178569418}\PY{p}{,}
                        \PY{n}{solver}\PY{o}{=}\PY{l+s+s2}{\PYZdq{}}\PY{l+s+s2}{liblinear}\PY{l+s+s2}{\PYZdq{}}\PY{p}{)}
\PY{n}{clf}\PY{o}{.}\PY{n}{fit}\PY{p}{(}\PY{n}{X\PYZus{}train}\PY{p}{,} \PY{n}{y\PYZus{}train}\PY{p}{)}
\end{Verbatim}
\end{tcolorbox}

            \begin{tcolorbox}[breakable, size=fbox, boxrule=.5pt, pad at break*=1mm, opacityfill=0]
\prompt{Out}{outcolor}{102}{\boxspacing}
\begin{Verbatim}[commandchars=\\\{\}]
LogisticRegression(C=0.20433597178569418, class\_weight=None, dual=False,
                   fit\_intercept=True, intercept\_scaling=1, l1\_ratio=None,
                   max\_iter=100, multi\_class='auto', n\_jobs=None, penalty='l2',
                   random\_state=None, solver='liblinear', tol=0.0001, verbose=0,
                   warm\_start=False)
\end{Verbatim}
\end{tcolorbox}
        
    \begin{tcolorbox}[breakable, size=fbox, boxrule=1pt, pad at break*=1mm,colback=cellbackground, colframe=cellborder]
\prompt{In}{incolor}{103}{\boxspacing}
\begin{Verbatim}[commandchars=\\\{\}]
\PY{c+c1}{\PYZsh{} Check coef\PYZus{} (coeficient)}
\PY{n}{clf}\PY{o}{.}\PY{n}{coef\PYZus{}}
\end{Verbatim}
\end{tcolorbox}

            \begin{tcolorbox}[breakable, size=fbox, boxrule=.5pt, pad at break*=1mm, opacityfill=0]
\prompt{Out}{outcolor}{103}{\boxspacing}
\begin{Verbatim}[commandchars=\\\{\}]
array([[ 0.00316728, -0.86044651,  0.66067041, -0.01156993, -0.00166374,
         0.04386107,  0.31275847,  0.02459361, -0.6041308 , -0.56862804,
         0.45051628, -0.63609897, -0.67663373]])
\end{Verbatim}
\end{tcolorbox}
        
    \begin{tcolorbox}[breakable, size=fbox, boxrule=1pt, pad at break*=1mm,colback=cellbackground, colframe=cellborder]
\prompt{In}{incolor}{104}{\boxspacing}
\begin{Verbatim}[commandchars=\\\{\}]
\PY{c+c1}{\PYZsh{} Match the coef\PYZsq{}s of features to columns}
\PY{n}{feature\PYZus{}dict} \PY{o}{=}  \PY{n+nb}{dict}\PY{p}{(}\PY{n+nb}{zip}\PY{p}{(}\PY{n}{df}\PY{o}{.}\PY{n}{columns}\PY{p}{,} \PY{n+nb}{list}\PY{p}{(}\PY{n}{clf}\PY{o}{.}\PY{n}{coef\PYZus{}}\PY{p}{[}\PY{l+m+mi}{0}\PY{p}{]}\PY{p}{)}\PY{p}{)}\PY{p}{)}
\end{Verbatim}
\end{tcolorbox}

    \begin{tcolorbox}[breakable, size=fbox, boxrule=1pt, pad at break*=1mm,colback=cellbackground, colframe=cellborder]
\prompt{In}{incolor}{105}{\boxspacing}
\begin{Verbatim}[commandchars=\\\{\}]
\PY{n}{feature\PYZus{}dict}
\end{Verbatim}
\end{tcolorbox}

            \begin{tcolorbox}[breakable, size=fbox, boxrule=.5pt, pad at break*=1mm, opacityfill=0]
\prompt{Out}{outcolor}{105}{\boxspacing}
\begin{Verbatim}[commandchars=\\\{\}]
\{'age': 0.0031672801993431563,
 'sex': -0.8604465072345515,
 'cp': 0.6606704082033799,
 'trestbps': -0.01156993168080875,
 'chol': -0.001663744504776871,
 'fbs': 0.043861071652469864,
 'restecg': 0.31275846822418324,
 'thalach': 0.024593613737779126,
 'exang': -0.6041308000615746,
 'oldpeak': -0.5686280368396555,
 'slope': 0.4505162797258308,
 'ca': -0.6360989676086223,
 'thal': -0.6766337263029825\}
\end{Verbatim}
\end{tcolorbox}
        
    \begin{tcolorbox}[breakable, size=fbox, boxrule=1pt, pad at break*=1mm,colback=cellbackground, colframe=cellborder]
\prompt{In}{incolor}{107}{\boxspacing}
\begin{Verbatim}[commandchars=\\\{\}]
\PY{c+c1}{\PYZsh{} Visualise feature importance }
\PY{n}{feature\PYZus{}df} \PY{o}{=} \PY{n}{pd}\PY{o}{.}\PY{n}{DataFrame}\PY{p}{(}\PY{n}{feature\PYZus{}dict}\PY{p}{,} \PY{n}{index}\PY{o}{=}\PY{p}{[}\PY{l+m+mi}{0}\PY{p}{]}\PY{p}{)}
\PY{n}{feature\PYZus{}df}\PY{o}{.}\PY{n}{T}\PY{o}{.}\PY{n}{plot}\PY{o}{.}\PY{n}{bar}\PY{p}{(}\PY{n}{title}\PY{o}{=}\PY{l+s+s2}{\PYZdq{}}\PY{l+s+s2}{feature importance}\PY{l+s+s2}{\PYZdq{}}\PY{p}{,} \PY{n}{legend}\PY{o}{=}\PY{k+kc}{False}\PY{p}{)}\PY{p}{;}
\end{Verbatim}
\end{tcolorbox}

    \begin{center}
    \adjustimage{max size={0.9\linewidth}{0.9\paperheight}}{output_92_0.png}
    \end{center}
    { \hspace*{\fill} \\}
    
    \begin{tcolorbox}[breakable, size=fbox, boxrule=1pt, pad at break*=1mm,colback=cellbackground, colframe=cellborder]
\prompt{In}{incolor}{109}{\boxspacing}
\begin{Verbatim}[commandchars=\\\{\}]
\PY{n}{pd}\PY{o}{.}\PY{n}{crosstab}\PY{p}{(}\PY{n}{df}\PY{p}{[}\PY{l+s+s2}{\PYZdq{}}\PY{l+s+s2}{sex}\PY{l+s+s2}{\PYZdq{}}\PY{p}{]}\PY{p}{,} \PY{n}{df}\PY{p}{[}\PY{l+s+s2}{\PYZdq{}}\PY{l+s+s2}{target}\PY{l+s+s2}{\PYZdq{}}\PY{p}{]}\PY{p}{)} \PY{c+c1}{\PYZsh{}look at the ratio}
\end{Verbatim}
\end{tcolorbox}

            \begin{tcolorbox}[breakable, size=fbox, boxrule=.5pt, pad at break*=1mm, opacityfill=0]
\prompt{Out}{outcolor}{109}{\boxspacing}
\begin{Verbatim}[commandchars=\\\{\}]
target    0   1
sex
0        24  72
1       114  93
\end{Verbatim}
\end{tcolorbox}
        
    \begin{tcolorbox}[breakable, size=fbox, boxrule=1pt, pad at break*=1mm,colback=cellbackground, colframe=cellborder]
\prompt{In}{incolor}{110}{\boxspacing}
\begin{Verbatim}[commandchars=\\\{\}]
\PY{n}{pd}\PY{o}{.}\PY{n}{crosstab}\PY{p}{(}\PY{n}{df}\PY{p}{[}\PY{l+s+s2}{\PYZdq{}}\PY{l+s+s2}{slope}\PY{l+s+s2}{\PYZdq{}}\PY{p}{]}\PY{p}{,} \PY{n}{df}\PY{p}{[}\PY{l+s+s2}{\PYZdq{}}\PY{l+s+s2}{target}\PY{l+s+s2}{\PYZdq{}}\PY{p}{]}\PY{p}{)}
\end{Verbatim}
\end{tcolorbox}

            \begin{tcolorbox}[breakable, size=fbox, boxrule=.5pt, pad at break*=1mm, opacityfill=0]
\prompt{Out}{outcolor}{110}{\boxspacing}
\begin{Verbatim}[commandchars=\\\{\}]
target   0    1
slope
0       12    9
1       91   49
2       35  107
\end{Verbatim}
\end{tcolorbox}
        
    \hypertarget{experimentation}{%
\subsection{6.Experimentation}\label{experimentation}}

if you haven't hit your evaluation metric yet\ldots{} ask
ypurself\ldots{}

\begin{itemize}
\tightlist
\item
  Could you collect more data?
\item
  Could you try a better model? CatBoost, XGBoost?
\item
  Could you improve the current models? (beyond what we've done so far)
\item
  If your model is good enough (you have hut ypure evaluation metric)
  how would ypu export it and share it with others?
\end{itemize}

    \begin{tcolorbox}[breakable, size=fbox, boxrule=1pt, pad at break*=1mm,colback=cellbackground, colframe=cellborder]
\prompt{In}{incolor}{ }{\boxspacing}
\begin{Verbatim}[commandchars=\\\{\}]

\end{Verbatim}
\end{tcolorbox}


    % Add a bibliography block to the postdoc
    
    
    
\end{document}
